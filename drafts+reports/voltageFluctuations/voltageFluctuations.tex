\documentclass[10pt]{article}
/Users/curandero/docs/mahvPreamble.tex

\begin{document}

\section{Model and analysis}

The presysnaptic spike trains were generated by 
\begin{eqnarray}
\partial_t p &=& p \frac{p_{\infty} - p}{\tau_p} + (1-p)
h \sum_{i}{\alpha(t-t_i)} 
\label{eq:dpdt}
\\
\partial_t x &=& x \frac{x_{\infty} - x}{\tau_x} - p x \sum_{i}{\alpha(t-t_i)} 
\label{eq:dxdt}
\\
\partial_t q &=& \frac{q_{\infty}(x,p) - q}{\tau_q}, \quad q \in
\lrSet{q_E, q_I}
\label{eq:dqdt}
\\
\partial_t v &=& - a_{\tNaK} \sinh\lrRound{
  \frac{v-v_{\tNaK}}{2v_T}}
- a_{E} q_E \sinh\lrRound{ \frac{v-v_{E}}{2v_T}}
- a_{I} q_I \sinh\lrRound{ \frac{v-v_{I}}{2v_T}}
\\
q_{\infty}(x,p) &=&  x p \sum_{i}{\alpha(t-t_i- 3\tau_{\alpha} )} 
\label{eq:dqdt}
\end{eqnarray}
with current amplitudes normalized by membrane capacitance.  The
parameters used for simulations can be found in
Table~\ref{table:parameters}.

\begin{figure}[h]
\includegraphics[width=0.8\textwidth]{figuresVoltageFluctuations/xpqv}
\caption{Simulation }
\end{figure}


\begin{table}[h]
\begin{small}
\caption{{Model parameters. All values for fast excitatory and inhibitory synaptic
    dynamics obtained from
    \cite{DestexheMainenSejnowski1994b,DestexheMainenSejnowski1998}
    and references therein.}
  The membrane capacitance was assumed to be similar to that of a
  compartment with either cylindrical or spherical shape having a radius
  of 5 $\mu$m, with a specific membrane capacitance of 0.0075 pF/$\mu^2$
  \citep{mainen1995model}. 
}\label{table:parameters}
\begin{tabular}{c r l  p{0.6\textwidth}}
\hline &&&
\\
Parameter & Value(s) & Units & Description 
\\
\hline &&&
\\
$C_m$ & $2.356$ & pF & Membrane capacitance of postsynaptic membrane
compartment 
\\
$\bar{a}_{\tNaK}$ & [1,10] & nA & Maximum amplitude for the current through {\natrium}-{\kalium} ATPase  
\\
$\bar{a}_{E}$ & [10.0,30.0] & pA & Maximum amplitude for the current mediated by
AMPA/kainate receptors  
\\
$\bar{a}_{I}$ & [1,3] & pA & Maximum amplitude for the current mediated by
GABA$_A$ receptors  
\\
$h$ & $[0,1]$ & -- &  Facilitation jump in the probability of release 
\\
$v_{\tNaK}$ & $-60$ & mV & Reversal potential for the {\natrium}-{\kalium} ATPase  
\\
$v_{E}$ & $0$ & mV & Reversal potential for fast, excitatory currents
mediated by AMPA/kinate receptors
\\
$v_{I}$ & $-70$ & mV & Reversal potential for fast, inhibitory currents
mediated by GABA$_A$ receptors
\\
$h$ & $[0,1]$ & -- &  Facilitation jump in the probability of release 
\\
$\tau_{\alpha}$ & $[0,10]$ & ms &  Time constant for the alpha
function describing the activation transients of synaptic release
\\
$\tau_{p}$ & $[10,200]$ & ms &  Return to baseline time constant for
the synaptic probability of release 
\\
$\tau_{x}$ & $[10,200]$ & ms &  Time constant for recovery from
synaptic depression
\\
$\tau_{A}$ & $\frac{1000}{1100+190} \approx 0.775$ & ms &  AMPA/kainate receptor time
constant 
\\
$\tau_{G}$ & $\frac{1000}{530+180} \approx 1.4 $& ms & GABA$_A$ receptor time
constant 
\\
\hline &&&
\\
\end{tabular}
\end{small}
\end{table}



\bibliographystyle{plainnat}
\bibliography{mahvPublications,membraneBiophysics,membraneBiophysics2}
\end{document}
