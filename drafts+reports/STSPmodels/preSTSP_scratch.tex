\section{Presynaptic release and short-term plasticity}

\paragraph{Binomial probability of release at the single synapse.} Assume the activation of the presynaptic machinery is a single event involving the independent release of $n$ vesicles in the  readily releasable pool (RRP) with a certain probability of release $p$. 
The expected number of vesicles released from each RRP is then $\bar{n}=np$. If the average quanta of neurotransmitter found in each vesicle is $q$, then the expected amount of neurotransmitter released for a given concentration of {\calcium} is $q n p$ and the maximum amount of neurotransmitter that can be released is $qn$. 

\paragraph{Time course of the probability of release at a single synapse.} The probability of release is not really fixed. The probability of release can be assumed to depend on the intracellular calcium ({\calcium}) concentration in the terminal, which depends on the opening of voltage-dependent channels and the clearance of calcium by intracellular buffers and intracellular machinery activated by {\calcium}. The time-dependent change in the  intracellular calcium concentration ($c$) in the presynaptic membrane can be modelled as \citep{AvRonParnasSegel1993} 
\begin{equation}
\partial_t c = \frac{c_m - c}{\tau_c}  - k_c I_{Ca}(v,c)
\label{eq:caDynamics}
\end{equation}
where $I_{Ca}$ is the total {\calcium}-current in the presynaptic terminal, and $v$ is the presynaptic membrane potential.  For simplification purposes, notice that the second term in equation \eqref{eq:caDynamics} can be replaced by time-dependent function that represents the increment in the {\calcium}-concentration as a function of time (e.g. a sum delta pulses or, alternatively, alpha functions, each activated at different times  representing the spike times). The probability of release at steady state can be then assumed to behave as 
\begin{equation}
p_{\infty}(c; c_p, g_p) = \frac{1}{1+ \exp\lrRound{ \frac{c_p -  c}{g_p}}}
\label{ssPRelease}
\end{equation}
where $c_p$ and $g_p$ are the half-concentration and the gain for the probability of release.  Given two values for the gain, $g_1$ and $g_2$ such that $g_{1}< g_{2}$, the corresponding probabilities of release are $p_1$ and $p_2$ are such that 
$p_1(c) > p_2(c)$ for $c<c_p$ and $p_1(c) < p_2(c)$ for $c>c_p$. 
The probability of release is more like an on/off switch for large enough values of $g_p$. Small enough values of $g_p$ yield leaky synapses. One interpretation of this behaviour is that the presynaptic release machinery with $g_1$ is more efficient than for $g_2$, for the same intracellular {\calcium}; an effect that could be produced by modulation. 

The probability of release at steady state is a monotonically increasing function. Therefore, if the concentration of {\calcium} increases, or the half-concentration for $p_{\infty}$, $c_p$ decreases, then the expected amount of neurotransmitter increases. In particular, if the {\calcium} dynamics are such that the firing frequency of the presynaptic neuron causes an accumulation

The steady state for $p$ can then be used to define the time-evolution of $p$ as 
\begin{equation}
\partial_t p = \frac{p_{\infty}(c) - p}{\tau_p}  
\label{eq:pRelDynamics}
\end{equation}

Let $q$ be the amount of neurotransmitter contained in the vesicles of the RRP, normalized by the maximum amount of neurotransmitter in the RRP. The time-dependent change in $q$ at time $t$ can be assumed to satisfy
\begin{equation}
\partial_t q = q \frac{q_{\infty} - q}{\tau_q} \lrRound{1- p}
\label{eq:ntRRPDynamics}
\end{equation}
The normalized amount of neurotransmitter released at time $t$ is then $p(t) q(t)$. 



\section{Somatic current as an average of synapses for each synaptic contact}
It is known that each posynaptic potential provokes a current $I_s(v)$ that   propagates toward the soma. This propagation entails a characteristic attenuation factor $\kappa$ due to morpholical properties that can be explained by a cable equation. The somatic current registered in experimental procedures can be seen as a sum of all synapses occurred at the same synaptic contact.
\begin{eqnarray}
I_s(v_i)=\sum_{i=1}^n\kappa_iI_i(v)
\end{eqnarray}
where $\kappa_i$ is the  attenuation factor correspondent to the $i$th synapse of the synaptic contact which could be assumed as the same for each $i$ if we suppose anatomic uniformity of the posynaptic cell.   
So 
\begin{eqnarray}
I_s(v_i)=\sum_{i=1}^n\kappa I_i(v)=\kappa \sum_{i=1}^n I_i(v)
\end{eqnarray}
Let it be $I_m(v)$ the average of the distribution of all synaptic currents of the same synaptic contact. Then $I_s(v_i)$ can be seen as a multiple of $\kappa I_m(v)$ 
\begin{eqnarray}
I_s(v_i)=\kappa \sum_{i=1}^n I_m(v)=n\kappa I_m(v)
\end{eqnarray}

\begin{appendix}

\section{Equations based on  the kinetic models by \cite{destexhe1998kinetic} }


\begin{eqnarray}
\partial_{t} c &=& \frac{c_{\infty}-c}{\tau_{c}} - k_{c} J_{Ca}, \quad \textrm{Intracellular [Ca]}
\\
\partial_{t} r &=& (1-r) k_{b} c^{4} - r  k_{u} , \quad \textrm{Calcium-dependent activation of release machinery}
\\
\partial_{t} w_{r} &=& w_{r} \lrRound{ \frac{w_{r\infty}-w_{r}}{\tau_{w}}} + k_{2}w_{l} -  k_{1}r~w_{r}, \quad \textrm{Readily Releasable Pool}
\\
\partial_{t} w_{l}&=& k_{1} ~r~w_{r}  - \lrRound{k_{3} +  k_{2}}w_{l}  , \quad \textrm{Active  vesicles}
\\
k_{3}~w_{l} 
& \propto & n, \quad \textrm{Released  vesicles}
\end{eqnarray}

\begin{table}
\begin{tabular}{r r l p{0.75\textwidth}}
Parameter & Value & Units & Description \\
\hline &&&\\
$c_{\infty}$ & 0.0001 & $\mu$M & Steady state concentration for intracellular calcium \citep{barroso2015diverse}\\
$\tau_{c}$ & 10-250 & ms & Time constant for the recovery intracellular calcium concentration \citep{barroso2015diverse} \\
$ k_{c}$ & --  &  & Impact of the voltage-dependent calcium current on the intracellular calcium \citep{barroso2015diverse} \\
$k_{b}$ & 10$^{20}$ & ($\mu$ M)$^{-4}$ ms$^{-1}$& Binding constant for the calcium-dependent activation of the release machinery \citep{DestexheMainenSejnowski1994b}\\
$k_{u}$ & 10$^{5}$ & ms$^{-1}$ & Unbinding constant for the calcium-dependent activation of the release machinery \citep{DestexheMainenSejnowski1994b}\\
$k_{1}$ & 10$^{9}$ & ms$^{-1}$ & Activation rate of RRP vesicles \citep{DestexheMainenSejnowski1994b}\\
$k_{2}$ & 10$^{5}$ & ms$^{-1}$ & Deactivation rate of RRP vesicles \citep{DestexheMainenSejnowski1994b}\\
$k_{3} $ & 4$\cdot 10^{6}$&  & Release rate of activated RRP vesicles  \\
$k_{n} $ & 4$\cdot 10^{7}$&  & Clearance rate for the neurotransmitter in the cleft \citep{DestexheMainenSejnowski1994b} \\
$w_{r\infty}$ &  &  & Steady state for the readily releasable pool \citep{barroso2015diverse} \\
$\tau_{w}$ & & ms  & Recovery time constant for the readily releasable pool \citep{barroso2015diverse} \\
\hline &&&\\
\end{tabular}
\end{table}


\end{appendix}
