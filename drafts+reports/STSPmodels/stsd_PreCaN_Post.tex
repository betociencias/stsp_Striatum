\documentclass[10pt]{article}
/Users/curandero/docs/mahvPreamble.tex

\begin{document}


\section{Short term presynaptic plasticity model}
Let represent an average $c_a = [Ca]_{i}$ in axonal terminals forming a synaptic contact. Let $p$ and $x$ represent the probability of release and the density of readily releasable vesicles in the terminal. A simple model for the dynamics of presynaptic release 
is then
\begin{eqnarray}
\partial_t c_a &=& \frac{c_{a\infty}-c_a}{\tau_c} - k I_N(v_{a})
\label{}
\\
p_{\infty}(c) &=& \frac{c^l}{c^l + {c_m}^l} 
\\
\partial_t p &=&  \frac{p_{\infty}(c)-p}{\tau_p} 
\\
\partial_t x &=& x \frac{x_{\infty}-x}{\tau_x} - p x h
%\\
%\partial_t u &=& u \frac{u_{\infty}-u}{\tau_u} - f(p(c_a) x)
\end{eqnarray}
where $h$ is equivalent to the time step in the time series, $l$ is the cooperativity of calcium for the increase in the probability of release.

\subsection{Analysis to predict asymptotic behaviours for periodic stimulation}

Assume $p \approx p_{\infty}(c)$ and write the surges in intracellular calcium as pulses given by delta functions. Equation~\eqref{}
\begin{eqnarray}
\partial_t c_a &=& \frac{c_{a\infty}-c_a}{\tau_c} - k \sum_{i=1}^n \delta(t-t_i) \\
p_{\infty}(c) &=& \frac{c^n}{c^n + {c_m}^n} 
\\
\partial_t x &=& x \frac{x_{\infty}-x}{\tau_x} - h p_{\infty}(c_a) x
\end{eqnarray}
Calcium dynamics: $c_{\infty} = 0 , c_0 = c(t_0) = k $
\\
Pulse times: $t_0, t_1, ..., t_n$
\begin{eqnarray*}
c_0 &=& c(t_0) = k
\\
c(t) &=& c_0 e^{-(t-t_0)/\tau_c}
=k e^{-(t-t_0)/\tau_c}, \quad t_0\leq t <t_1
\\
c_1 &=& c(t_1) = k \lrRound{1 + e^{-(t_1-t_0)/\tau_c}}
\\
c(t) &=& c_1 e^{-(t-t_1)/\tau_c}, \quad t_1\leq t <t_2
\\
c_2 &=& k+ c_1 e^{-(t_2-t_1)/\tau_c}  
\\
&=& k+ k \lrRound{1 + e^{-(t_1-t_0)/\tau_c}} e^{-(t_2-t_1)/\tau_c} 
\\
&=& k+ k  \lrRound{e^{-(t_2-t_1)/\tau_c}  + e^{-(t_2-t_0)/\tau_c}} 
\\
&=& k \lrRound{1+e^{-(t_2-t_1)/\tau_c}  + e^{-(t_2-t_0)/\tau_c}}
\end{eqnarray*}
Therefore, after the $n$th pulse has arrived,
\begin{eqnarray}
c_n &=& k \lrRound{ 
\sum_{i=0}^{n} e^{-(t_n-t_{i})/\tau_c} 
}
\label{eq:cn}
\\
c(t) &=& c_n e^{-(t-t_n)/\tau_c}, \quad t_n\leq t <t_{n+1}
\label{eq:c(t)tn}
\end{eqnarray}
If pulses are periodic (fixed frequency), with $\delta = t_{i+1}-t_i$, then $t_n - t_0 = n \delta$, and equation~\eqref{eq:cn} transforms into
\begin{eqnarray}
c_n &=& k \lrRound{ 
\sum_{i=0}^{n} e^{-i\delta/\tau_c} 
}
\label{eq:cnPeriodicPulses}
\end{eqnarray}
The quantity $a = e^{-\delta/\tau_c}$ represents the calcium decay factor between pulses. Then the calcium concentration peak after the $n$th pulse is 
\begin{equation}
c_n=k \lrRound{ 
\sum_{i=0}^{n} a^{i} 
}
= k \lrRound{\frac{1-a^{n+1}}{1-a}}
= k \lrRound{\frac{1-e^{-\lrRound{n+1} \delta/\tau_c}}{1-e^{\lrRound{- \delta/\tau_c}}}}
\end{equation}
and the calcium concentration right before the $(n+1)$th pulse is 
\begin{equation}
\lim_{t\rightarrow t_{n+1}} c(t)
= k \lrRound{\frac{1-e^{-\lrRound{n+1} \delta/\tau_c}}{1-e^{\lrRound{- \delta/\tau_c}}}} e^{-\delta/\tau_c}
\end{equation}
For instance, if $\delta=50$ ms and $\tau_c = 10$ ms, then $e^{-\delta/\tau_c}=e^{-5}\approx 0.0067$, which means that there is almost no accumulation of intracellular calcium between pulses at 20 Hz. 

Once the time course of $c$ determined, it is possible to predict $p$ and $x$. For instance, the probability of release right after the $n$th pulse is,
\begin{eqnarray}
p_{\infty}(c_n) &=& \frac{c_n^l}{c_n ^l + {c_m}^l} 
\end{eqnarray}
which allows us to calculate the solution for $x$ recursively by considering initial conditions defined by the pulses, as done with $c$. 

Explicitly, assume $x(t) = x_{\infty}$ for $t<t_0$.  The value of $x$ after the pulse at time $t_0$ jumps to 
$$
x_0 = x_{\infty} - p x h = x_{\infty} - x h \frac{c_0^l}{c_0^l + {c_m}^l}. 
$$
Then, 
\begin{eqnarray}
x(t) &=& \frac{x_{\infty}}{1 + \frac{x_{\infty}- x_0}{x_0}\exp\lrRound{-\frac{x_{\infty}}{\tau_x} (t-t_0)}}
\end{eqnarray}
for $t_0\leq t < t_1$. Similarly, when the next pulse arrives at time $t=t_1$, the value of $x$ jumps to 
\begin{eqnarray}
x_1= \frac{x_{\infty}}{1 + \frac{x_{\infty}- x_0}{x_0}\exp\lrRound{-\frac{x_{\infty}}{\tau_x} (t_1-t_0)}} - x h \frac{c_1^l}{c_1 ^l + {c_m}^l}
\end{eqnarray}
and the dynamics between the pulses at times $t_1$ and $t_2$ are then
\begin{eqnarray}
x(t) &=& \frac{x_{\infty}}{1 + \frac{x_{\infty}- x_1}{x_1}\exp\lrRound{-\frac{x_{\infty}}{\tau_x} (t-t_1)}}
\end{eqnarray}

\end{document}