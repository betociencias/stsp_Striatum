% !TEX root = stsp3d2d_main.tex
\section{Reduced models of short term synaptic plasticity}

Let $c$, $p$, and $x$ represent the intracellular concentration of {\calcium} at the presynaptic terminal, the proportion of released vesicles per unit time, and the normalized readily releasable vesicles. Assume that the dynamics are given by
\begin{eqnarray}
\partial_{t} c &=& \frac{c_{b}-c}{\tau_{b}} - k_{r} \alpha_{r}(1-r)c + f(t)
\label{dc}
\\
\partial_{t} r &=&  \alpha_{r}~c~(1-r) - \beta_{r} r  
\label{dr}
\\
\partial_{t} q &=& \frac{q_{\infty}-q}{\tau_{q}} - \delta~q~r
\label{dq}
\end{eqnarray}
where $c_{b}$ and $\tau_{b}$ represent the steady state and time constant for presynaptic calcium due to buffering in the absence of any other perturbation. The term $f(c)$ represents the flux of \calcium into the terminal via voltage-dependent channels. The term $k_{c} \alpha_{r}~c~(1-r)$ represents the decrease in the calcium concentration due to binding to the release machinery.

\begin{table}[h]
\centering
\caption{Parameters and values}
\begin{tabular}{l l l p{0.75\textwidth}}
\hline 
$c_{b}$ & & $\mu$M & steady state constant for presynaptic {\calcium} due to buffering
\\
$\alpha_{r}$ & & ms$^{-1}$ $\mu$M$^{-1}$ & Activation rate of the release machinery in the presence of {\calcium}
\\
$\beta_{r}$ & & ms$^{-1}$ & Deactivation rate of the release machinery
\\
$k_{r}$ & & $\mu$M & Impact of the {\calcium} activating the release machinery on the intracellular {\calcium} concentration in the terminal
\\
$q_{\infty}$ & &  & Steady state for que normalized quanta in the readily releasable pool
\\
$\tau_{b}$ & & ms &  time constant for presynaptic {\calcium} due to buffering
\\
$\tau_{q}$ & & ms & Time constant for the recovery of the readily releasable pool
\\
$\delta$ & & ms$^{-1}$ & Conversion rate 
\\
\hline 
\end{tabular}
\end{table}


% ------------------------------------
\subsection{{\calcium} dynamics}
% ------------------------------------

It is worth noticing that the dynamics for $c$ are linear when $f(t)=0$. To see this, equation~\eqref{dc} can be transformed to 
\begin{eqnarray*}
\partial_{t} c 
%&=& f(t) +\frac{c_{b}-c}{\tau_{b}} - k_{r} \alpha_{r}~c~(1-r) + f(t)
%\\ &=& f(t) +\frac{c_{b}}{\tau_{b}} - c \lrRound{ \frac{1}{\tau_{b}} + k_{r} \alpha_{r}~(1-r)} 
%\\ &=& f(t) +\frac{c_{b}}{\tau_{b}} - c \lrRound{ \frac{1+ k_{r} \alpha_{r}~(1-r) \tau_{b} }{\tau_{b}} } 
%\\ 
&=& f(t) + \frac{c_{\infty}-c}{\tau_{c}} 
\label{dc2}
\end{eqnarray*}
where 
\begin{eqnarray}
\tau_{c} &=& \frac{\tau_{b}}{1+ k_{r} \alpha_{r}~(1-r) \tau_{b} },
\\
c_{\infty} &=& \frac{c_{b}}{1+ k_{r} \alpha_{r}~(1-r) \tau_{b} }.
\end{eqnarray}
That is, if $f(t)=0$, then the dynamics for $c$ from an initial condition $c_{o}=c(t_{o})$ behave like 
$$c(t) = c_{\infty} - \lrRound{c_{\infty}- c_{0}}\exp \lrRound{\frac{t_{o}-t}{\tau_{c}}}.$$
which means that in the absence of any {\calcium} influx, the concentration of {\calcium} in the presynaptic terminal goes to a steady state $c_{\infty}$ that becomes the steady state for buffering only for fast buffering (small $\tau_{b}$), or if the rate of binding to the activation machinery is slow enough (small $\alpha_{r}$). 

\paragraph{Calcium influx. }
To better understand the {\calcium} dynamics, this time in the presence of calcium influx, 
assume $f(t)$ is a Dirac comb given by 
\begin{equation}
f(t) =  h ~\sum_{k=1}^{n} \delta\lrRound{t-t_{k}}
\end{equation}
where $t_{0}$, ..., $t_{n}$ represent stimulus times (e.g. presynaptic action potential-driven fluxes).  Also, suppose that there has been a long enough interval of time so that before the first pulse, presynaptic terminal {\calcium} concentration is at steady state; that is, that $c(t)=c_{\infty}$ for $t\leq t_{0}$. Also, suppose that, at the $k$th pulse time, the value of $c(t)$ changes to $c_{k}= c(t_{k})+ h$, for $k \in \lrSet{0,...,n}$. 
Then, at $t=t_{0}$, let $c_{0} = c_{\infty} + h$ be the new value of $c$ after the jump. The dynamics for $c$ before the next pulse are given by, 
\begin{eqnarray*} 
c(t) %&=& c_{\infty} - \lrRound{c_{\infty}-c_{0}}\exp\lrRound{ \frac{t_{0}-t}{\tau_{c}}}
%\\
%&=& c_{\infty} - \lrRound{c_{\infty}- c_{\infty} -h}\exp\lrRound{ \frac{t_{0}-t}{\tau_{c}}}
%\\
&=& 
c_{\infty} + h\exp\lrRound{ \frac{t_{0}-t}{\tau_{c}}}
\end{eqnarray*} 
for $t \in (t_{0},t_{1})$. At $t=t_{1}$ the calcium concentration changes again, to $c_{1}= c(t_{1})+ h$, which becomes a new initial condition. Then, for $t \in (t_{1},t_{2})$, 
\begin{eqnarray*} 
c(t) &=& c_{\infty} - \lrRound{c_{\infty}-c_{1}}\exp\lrRound{ \frac{t_{1}-t}{\tau_{c}}},
%\\
%&=& c_{\infty} - \lrRound{
%c_{\infty}-  c_{\infty}  - h \exp\lrRound{ \frac{t_{0}-t_{1}}{\tau_{c}}} - h} \exp\lrRound{ \frac{t_{1}-t}{\tau_{c}}},
%\\
%&=& c_{\infty} + h\lrRound{ 1+ \exp\lrRound{ \frac{t_{0}-t_{1}}{\tau_{c}}}}\exp\lrRound{ \frac{t_{1}-t}{\tau_{c}}},
\\ 
&=& c_{\infty} + h\lrRound{  \exp\lrRound{ \frac{t_{1}-t}{\tau_{c}}} + \exp\lrRound{ \frac{t_{0}-t}{\tau_{c}}}}.
 \end{eqnarray*} 
In general, for $t \in (t_{n-1},t_{n})$,
 \begin{eqnarray*} 
c(t) &=&  c_{\infty} +h \sum_{k=0}^{n-1} \exp\lrRound{ \frac{t_{k}-t}{\tau_{c}}}
 \end{eqnarray*} 
At $t_{n}$ the value of $c$ jumps to
 \begin{eqnarray*} 
c(t_{n}) &=&  c_{\infty} +h \sum_{k=0}^{n} \exp\lrRound{ \frac{t_{k}-t_{n}}{\tau_{c}}}
 \end{eqnarray*}
 If pulses are periodic, with $d= t_{k+1}-t_{k}$, for all $k \in \lrSet{1,...,n}$ then $t_{n}-t_{k} =(n-k) d$.  
 In this case, the calcium concentration becomes
 \begin{eqnarray*} 
c(t_{n}) &=&  c_{\infty} +h  \sum_{k=0}^{n} u^{\lrRound{ n-k}} =  c_{\infty} +h  \sum_{l=0}^{n} u^{l},
\\
&=&  c_{\infty} +h \lrRound{ \frac{1-u^{n+1}}{1-u} },
\end{eqnarray*}
where
 \begin{equation}
 u= \exp\lrRound{- \frac{d}{\tau_{c}}}.
 \end{equation}

 
Explicitly,
 \begin{eqnarray} 
c(t_{n}) &=&    c_{\infty} +h \lrSquare{ \frac{1-\exp\lrRound{- (n+1)\frac{d}{\tau_{c}}}}{1-\exp\lrRound{- \frac{d}{\tau_{c}}}} }.
\end{eqnarray}
The asymptotic behavior as $n\rightarrow \infty$ is then 
\begin{equation} 
c_{*}  =  c_{\infty} +h \lrRound{ \frac{1}{1-\exp\lrRound{- \frac{d}{\tau_{c}}}} }.
\end{equation}
which written explicitly in terms of the parameters for {\calcium} buffering and activation of the release machinery, becomes
\begin{equation} 
c_{*}  =  c_{\infty} +h \lrRound{ \frac{1}{1-\exp\lrRound{- d\frac{1+ k_{r} \alpha_{r}~(1-r) \tau_{b} }{\tau_{b}}}} }.
\end{equation}

% ---------------------------------------
\subsection{$p$ dynamics and the subsystem $c-p$}
% ---------------------------------------
The dynamics of the subsystem $c,p$ are given by 
\begin	{eqnarray}
\partial_{t} c &=& \frac{c_{\infty}-c}{\tau_{c}} + f(t)
\\
\partial_{t} p &=&  \alpha ~c~(1-p) - \beta p  
\end{eqnarray}
The time constant and steady state for the $p$ are 
\begin	{eqnarray}
\tau_{p}(c) &=& \frac{1}{\alpha ~c + \beta} 
\\
p_{\infty}(c) &=&  \frac{\alpha ~c}{\alpha ~c + \beta} = \frac{c}{c + \frac{\beta}{\alpha}}
\end{eqnarray}
$c$ increases as action potentials arrive, then $\tau_{p}(c)$  decreases and $p_{\infty}(c)$ increases. So $p$ increases with $c$, but the dynamics for $p$ become faster as that happens. 

if $\tau_{c}>>\tau_{p}$, then 
the dynamics for $p$ are fast enough to substitute $p$ with $p_{\infty}(c)$. Explicitly, it would be required that 
\begin	{equation*}\tau_{c}>\frac{1}{\alpha ~c + \beta}.
\end{equation*}




