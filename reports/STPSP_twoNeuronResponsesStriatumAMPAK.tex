\documentclass[10pt]{article}
%%%%%%%%%%%%%%%%%%%%%%%%%%%
% Packages
%%%%%%%%%%%%%%%%%%%%%%%%%%%
\usepackage{hyperref}
\hypersetup{
  pdfauthor={Marco Arieli Herrera-Valdez},
  pdftitle={}
  pdftex,
  colorlinks=true,
  urlcolor=Bittersweet,
  linkcolor=blue,
  pdftoolbar=true,
  pdfmenubar=true,
  citecolor=Purple,
  filecolor=blue, 
}
%\usepackage[spanish,es-nodecimaldot]{babel}
\usepackage{lscape}
\usepackage{setspace} 
%\setstretch{1.1}
%\doublespacing
\usepackage[utf8]{inputenc}
%\usepackage[latin1]{inputenc}
%\usepackage[applemac]{inputenc}
%% Only if the base font of the document is to be different, say sans serif
% Text layout
\usepackage[T1]{fontenc}
\usepackage[scaled=0.92]{helvet}
\renewcommand*\familydefault{\sfdefault}
\usepackage{authblk}
\usepackage{graphicx}
\usepackage{python}
\usepackage{mathtools,amsfonts,amssymb,amsmath}
\usepackage[dvipsnames,svgnames,hyperref,table]{xcolor}
\usepackage{siunitx}
%\usepackage{xcolor}
\usepackage{microtype}
\usepackage{sidecap}
%\hypersetup{pdfpagemode=FullScreen}
%\usepackage[left=2.5cm,right=2.5cm,top=0cm,bottom=2cm,includehead]{geometry}
\usepackage{geometry}
\usepackage{fancyhdr}
%\pagestyle{fancyplain}
%\pagestyle{plain}
% The color packages must appear before the pdfpages package
%\usepackage{chemarr}
\usepackage{listings}
%\usepackage[normalem]{ulem}
%\usepackage[usenames,svgnames,dvipsnames]{xcolor}
%\usepackage[dvipsnames,svgnames,usenames]{xcolor}
%\usepackage{booktabs} % Top and bottom rules for table
\usepackage[font=small,labelfont=bf]{caption} 
\usepackage{wrapfig}
%\usepackage{subfigure}
%\usepackage{beamerthemesplit}
\usepackage{boldline}
\usepackage{multirow}
\usepackage{multicol}
\usepackage{longtable}
\usepackage{times}
\usepackage{animate}
\usepackage{pdfpages}
\usepackage{url}
%\usepackage{multimedia}
%\usepackage{movie15} 
%\usepackage{media9} 
\usepackage{verbatim}
%\usepackage{pgflibraryarrows} 
%\usepackage{pgflibraryshapes}
\usepackage{tikz}
%\usetikzlibrary{arrows,shapes,matrix,chains,calc,positioning} 
%\usetikzlibrary{trees,mindmap} 
\usepackage{ifthen}
\usepackage{animate}

% To get the envelope in the author list
\usepackage[misc]{ifsym}
%\usepackage[misc,geometry]{ifsym}

%\Letter after the name of the corresponding author

% --------------------------------------
% Bibliography
% --------------------------------------
%\usepackage[sort&compress]{natbib} 
\usepackage[round,sort&compress]{natbib} 
%\usepackage[numbers,sort&compress]{natbib} 
%\bibliographystyle{plainnat}



% ------------------------------------------------
% Abbreviations and other commands
% ------------------------------------------------
\newcommand{\email}{\textrm{\Letter}}
\newcommand{\corrAuthor}{\textrm{\Letter}}
% ------------------------------------------------
\newcommand{\figref}[1]{{Fig.~\ref{#1}}}
\newcommand{\twofigrefs}[2]{{Figs.~\ref{#1}~and~\ref{#2}}}
\newcommand{\figrefs}[2]{{Figs.~\ref{#1}-\ref{#2}}}
\newcommand{\eqnref}[1]{{Eq.~\eqref{#1}}}
\newcommand{\twoeqnrefs}[2]{{Eqs.~\eqref{#1}~and~\eqref{#2}}}
\newcommand{\eqnrefs}[2]{{Eqs.~\eqref{#1}-\eqref{#2}}}
\newcommand{\eqn}[1]{\begin{equation}#1\end{equation}}
\newcommand{\eqna}[1]{\begin{eqnarray}#1\end{eqnarray}}
\newcommand{\smalleq}[1]
{\begin{equation}{\small #1}\end{equation}}
\newcommand{\smalleqna}[1]{ 
  \begin{small} 
    \begin{eqnarray} #1 \end{eqnarray} 
  \end{small}
}
\newcommand{\tinyeqna}[1]{ 
  \begin{tiny} 
    \begin{eqnarray} #1 \end{eqnarray} 
  \end{tiny}
}
\newcommand{\lrRound}[1]{\left(#1\right)}
\newcommand{\lrSquare}[1]{\left[#1\right]}
\newcommand{\lrSet}[1]{\left\{#1\right\}}
\newcommand{\lrAbs}[1]{\left|#1\right|}
\newcommand{\lrNorm}[1]{\left\|#1\right\|}
\newcommand{\prob}[1]{\mathbf{P}\left\{ #1\right\}}
\newcommand{\eg}{\textit{e.g.}}
\newcommand{\ie}{\textit{i.e.}}
\newcommand{\celsius}{{$^o$}}
\newcommand{\potassium}{{K$^+$}}
\newcommand{\kalium}{{K$^+$}}
\newcommand{\hydrogen}{{H$^+$}}
\newcommand{\sodium}{{Na$^+$}}
\newcommand{\natrium}{{Na$^+$}}
\newcommand{\calcium}{{Ca$^{2+}$}}
\newcommand{\chloride}{{Cl$^{-}$}}
\newcommand{\magnessium}{{Mg$^{2+}$}}
\newcommand{\concNa}{[Na]}
\newcommand{\concCa}{[Ca]}
\newcommand{\concCl}{[Cl]}
\newcommand{\concK}{[K]}
\newcommand{\felis}{{{$I_{cyc}$}}}
\newcommand{\icyc}{{{$I_{cyc}$}}}
\newcommand{\Avogadro}{N_{\mathrm{A}}}
\newcommand{\absTemp}{\mathrm{T}}
\newcommand{\GasConstant}{\mathrm{R}}
\newcommand{\Faraday}{\mathrm{F}}
\newcommand{\hPlanck}{\mathrm{h}}
\newcommand{\kBoltzmann}{\mathrm{k}}
\newcommand{\kT}{\mathrm{kT}}
\newcommand{\qElementary}{\mathrm{q}}
\newcommand{\hqovertwokT}[1]{\frac{\mathrm{e_0}#1}{2\mathrm{kT}}}
\newcommand{\hqoverkT}[1]{\frac{\mathrm{e_0}#1}{\mathrm{kT}}}
\def\qovertwokT{\mathrm{\frac{e_0}{2kT}}}
\def\qoverkT{\mathrm{\frac{e_0}{kT}}}
\newcommand{\trace}[1]{{\mathrm{Tr}(#1)}}
\newcommand{\tSm}{{\mathrm{Sm}}}
\newcommand{\tSy}{{\mathrm{Sy}}}
\newcommand{\tSt}{{\mathrm{St}}}
\newcommand{\tF}{{\mathrm{F}}}
\newcommand{\tLFP}{{\mathrm{LFP}}}
\newcommand{\tATP}{{\mathrm{ATP}}}
\newcommand{\tADP}{{\mathrm{ADP}}}
\newcommand{\tCaATP}{{\mathrm{CaATP}}}
\newcommand{\tP}{{\mathrm{P}}}
\newcommand{\tNa}{{\mathrm{Na}}}
\newcommand{\tNaK}{{\mathrm{NaK}}}
\newcommand{\tNaKa}{{\mathrm{NaKa}}}
\newcommand{\tNaP}{{\mathrm{NaP}}}
\newcommand{\tNaT}{{\mathrm{NaT}}}
\newcommand{\tNaH}{{\mathrm{NaH}}}
\newcommand{\tNaCa}{{\mathrm{NaCa}}}
\newcommand{\tNaKCl}{{\mathrm{NaKCl}}}
\newcommand{\tKCl}{{\mathrm{KCl}}}
\newcommand{\tCa}{{\mathrm{Ca}}}
\newcommand{\tCl}{{\mathrm{Cl}}}
\newcommand{\tCaL}{{\mathrm{CaL}}}
\newcommand{\tE}{{\mathrm{E}}}
\newcommand{\tExtra}{{\mathrm{Extra}}}
\newcommand{\tExt}{{\mathrm{Ext}}}
\newcommand{\tI}{{\mathrm{I}}}
\newcommand{\tSyn}{{\mathrm{Syn}}}
\newcommand{\tIn}{{\mathrm{In}}}
\newcommand{\tAK}{{\mathrm{AK}}}
\newcommand{\tGA}{{\mathrm{GA}}}
\newcommand{\tGB}{{\mathrm{GB}}}
\newcommand{\tGlu}{{\mathrm{GluA}}}
\newcommand{\tAMPA}{{\mathrm{AMPA}}}
\newcommand{\tGABA}{{\mathrm{GABA}}}
\newcommand{\tGabaA}{{\mathrm{GabaA}}}
\newcommand{\tACh}{{\mathrm{ACh}}}
\newcommand{\tS}{{\mathrm{S}}}
\newcommand{\tDK}{{\mathrm{DK}}}
\newcommand{\tSK}{{\mathrm{SK}}}
\newcommand{\tH}{{\mathrm{H}}}
\newcommand{\tK}{{\mathrm{K}}}
\newcommand{\tKaD}{{\mathrm{KaD}}}
\newcommand{\tKD}{{\mathrm{KD}}}
\newcommand{\tL}{{\mathrm{L}}}
\newcommand{\INa}{{I_\mathrm{Na}}}
\newcommand{\INaT}{{I_\mathrm{NaT}}}
\newcommand{\ICa}{{I_{\mathrm{Ca}}}}
\newcommand{\ICaL}{{I_{\mathrm{CaL}}}}
\newcommand{\IKD}{{I_{\mathrm{KD}}}}
\newcommand{\IKaD}{{I_{\mathrm{KaD}}}}
\newcommand{\IK}{{I_{\mathrm{K}}}}
\newcommand{\IAMPA}{{I_{\mathrm{AMPA}}}}
\newcommand{\INMDA}{{I_{\mathrm{NMDA}}}}
\newcommand{\IGABA}{{I_{\mathrm{GABA}}}}
\newcommand{\INaKATP}{{I_{\mathrm{NaK}}}}
\newcommand{\INaK}{{I_{\mathrm{NaK}}}}
\newcommand{\INaKa}{{I_{\mathrm{NaKa}}}}	
\newcommand{\INaCaX}{{I_{\mathrm{NaCa}}}}
\newcommand{\ILeak}{{I_{\mathrm{L}}}}
\newcommand{\stimtop}{{\textit{Top}}}
\newcommand{\stimup}{{\textit{Up}}}
\newcommand{\stimdown}{{\textit{Down}}}
\newcommand{\diffop}[2]{{\partial_{#1} #2}}
%%%%%%%%%%%%%%%%%%%%%%%%%%%
\newcommand{\artTitle}[3]{
\begin{flushleft}
\begin{Large}
{#1}
\end{Large} \\ \smallskip
\begin{large}
{#2}
\end{large} \\ \smallskip
{#3}
\end{flushleft}
}
%%%%%%%%%%%%%%%%%%%%%%%%%%%
\newcommand{\splitPage}[4]{
\begin{minipage}{#1\textwidth}
{#2}
\end{minipage}%
\begin{minipage}{#3\textwidth}
{#4}
\end{minipage}
}
%
% LaTeX beamer
\newcommand{\slide}[1]{\begin{frame}#1\end{frame}}
\newcommand{\from}[1]{{\tiny From #1.}}
\newcommand{\fuente}[1]{{\tiny #1}}
\newcommand{\source}[1]{{\tiny #1}}
\newcommand{\sS}[3]{{{#1}_{#2}^{#3}}}


%%%%%%%%%%%%%%%%%%%%%%%%%%%
% Edits and reviews
%%%%%%%%%%%%%%%%%%%%%%%%%%%
\newcommand{\aquimequede}{
\vspace{1cm} \textcolor{red}
{\textbf{AQUI ME QUEDE}} \\
\vspace{1cm}}
\newcommand{\mahv}[1]{\textcolor{gray}{$^{MAHV:}$}\textcolor{orange}{#1} }



%%%%%%%%%%%%%%%%%%%%%%%%%%%
% Colored sections
%%%%%%%%%%%%%%%%%%%%%%%%%%%
\newcommand{\headcolor}[1]{\textcolor{Bittersweet}{#1}}
\newcommand{\colorsection}[1]{\section*{\headcolor{#1}}}
\newcommand{\colorsubsection}[1]{\subsection*{\headcolor{#1}}}
\newcommand{\colorparagraph}[1]{\paragraph{\headcolor{#1}}}
\definecolor{lightGray}{rgb}{0.92,0.92,0.92}


%%%%%%%%%%%%%%%%%%%%%%%%%%%
% Figure settings and inclusion/exclusion of figures from the compilation
%%%%%%%%%%%%%%%%%%%%%%%%%%%

\newcommand{\figu}[3]{
\begin{figure}{\begin{center}#3\end{center}}\caption{#1}\label{#2}\end{figure}}
\newcommand{\tablFigu}[3]{
\begin{table}\caption{#1}\label{#2}\begin{center}{#3}\end{center}\end{table}}
%
\newcommand{\smallcaption}[1]{\begin{small}\caption{{#1}}\end{small}}
%\renewcommand{\caption}[1]{\caption{\begin{small}#1\end{small}}}
\usepackage{comment}
%\excludecomment{cond}
\includecomment{cond}
\newcommand{\showfigs}[1]{}
\begin{cond}
\renewcommand{\showfigs}[1]{\begin{center}#1\end{center}}
\end{cond}

% --------------------------------------
% mahvMacros
% --------------------------------------
% For supplements
% --------------------------------------
%\newcommand{\hbAppendixPrefix}{A}
%
%\renewcommand{\thefigure}{\hbAppendixPrefix\arabic{figure}}
%\setcounter{figure}{0}
%\renewcommand{\thetable}{\hbAppendixPrefix\arabic{table}} 
%\setcounter{table}{0}
%\renewcommand{\theequation}{\hbAppendixPrefix\arabic{equation}} 
%\setcounter{equation}{0}


\newenvironment{suppFig}[4]{
  \begin{figure}[h]
    \begin{center}
      {\includegraphics[#4]{#3}} \caption{\small #1} \label{#2}
    \end{center}
  \end{figure}}

\newenvironment{tab}[3]{
\begin{table}
\begin{center}
{#3} \caption{\small #1} \label{#2}
\end{center}
\end{table}}

% Operators
\DeclareMathOperator{\Tr}{Tr}
\newcommand{\re}[1]{ {\mathfrak{Re}} \left( {#1} \right)}
\newcommand{\im}[1]{ {\mathfrak{Im}} \left( {#1} \right)}
% Transpose upperscript
\newcommand{\transp}[1]{#1^{\mathsmaller T}}

% Column vectors. Usage:
% \colvector{a\\b\\c\\d\\e}
\newcommand{\colvector}[1]{\begin{pmatrix}#1\end{pmatrix}} 

% Column vectors. Usage:
% \colvec{5}{a}{b}{c}{d}{e}
\newcount\colveccount
\newcommand*\colvec[1]{
        \global\colveccount#1
        \begin{pmatrix}
        \colvecnext
}
\def\colvecnext#1{
        #1
        \global\advance\colveccount-1
        \ifnum\colveccount>0
                \\
                \expandafter\colvecnext
        \else
                \end{pmatrix}
        \fi
}

% ----------------------------------
% Code listing 
% ----------------------------------
%\lstdefinestyle{customPython}{
%  language=Python,
%  belowcaptionskip=1\baselineskip,
%  breaklines=true,
%  %frame=L,
%  xleftmargin=\parindent,
%  showstringspaces=false,
%  basicstyle=\footnotesize\ttfamily,
%  keywordstyle=\bfseries\color{green!40!black},
%  commentstyle=\itshape\color{purple!40!black},
%  identifierstyle=\color{blue},
%  stringstyle=\color{orange},
%}

%\newcommand<>{\hover}[1]{\uncover#2{%
% \begin{tikzpicture}[remember picture,overlay]%
% \draw[fill,opacity=0.4] (current page.south west)
% rectangle (current page.north east);
% \node at (current page.center) {#1};
% \end{tikzpicture}}
%}

\setstretch{1.2}
%
\begin{document}
%
\def\Titulo{A model of short term presynaptic plasticity between pairs of connected striatal neurons in response to varying degrees of common fast glutamatergic inputs }
\def\Autores{Marco Arieli Herrera-Valdez$^{1,\email}$}
\def\Afiliaciones{$^{1}$ Laboratorio de Fisiología de Sistemas, Facultad de Ciencias, Universidad Nacional Autónoma de México}
\def\correosE{marcoh@ciencias.unam.mx}
\artTitle{\Titulo}{\Autores}{\Afiliaciones}{\correosE}
% 
\begin{abstract}
A model to study short-term synaptic plasticity as observed in principal and local interneurons in the rodent striatum in response to glutamatergic synaptic input is presented. 
\end{abstract}
%
\section{Mathematical modeling of striatal neurons}

The model dynamics are given by three ordinary differential equations that describe the time-dependent changes of $v$, $w$, and $c$, respectively representing the transmembrane potential, the proportion of open delayed rectifier K$^+$ channels, and the intracellular {\calcium} concentration \citep{herrera2018thermodynamic}.

The change in the membrane potential is the sum of the transmembrane ionic fluxes normalized by the membrane capacitance. Explicitly,
\begin{equation}
C_m \partial_t v= I_{\tF} -I_{\tIn}-  I_{\tSyn}, \label{eq:dvdt}
\end{equation}
with 
\begin{eqnarray}
I_{\tSyn} &=&  I_{\tGA}(v,u) + I_{\tAK}(v,c), \label{eq:Isyn} \\
I_{\tIn} &=& I_{\tNaK}(v) + I_{\tNaT}(v,w) + I_{\tCaL}(v,c) + I_{\tDK}(v,w) + I_{\tSK}(v,c)  \label{eq:Iin}
\end{eqnarray}

Here $\partial_t G$ represents the instantaneous change in G with respect to time. $C_m$ (pF) is a constant representing the change in charge around the membrane with respect to the membrane potential typically referred to as membrane capacitance in conductance-based models, \cite{herrera2020nonequivalent}). The term $I_{F}$ represents a stimulus \textit{forcing} the membrane either by incoming current from an electrode, or from the local field potential (simulations of spontaneous activity). 
The fluxes in quation~\eqref{eq:dvdt} are all given by the product of an amplitude term~(pA), a gating term,  a flux driving force (Table~\ref{tab:ionFluxes}). The amplitude terms $a_x$  are given by $s_x N_x$. The term
$s_x$ (pA) is the current flowing through a single transmembrane protein (typically around 1 pA for most voltage-gated channels \citep{hille2001ion}), and $N_x$ is the number of membrane proteins mediating the current (e.g. {number of \kalium channels}). In our calculations and estimations of the contributions of the different ion fluxes to the change in $v$, we use $a_x = s_x N_x$ (pA) as an approximation for the whole-cell current.  
The flux across the membrane mediated by the different transmembrane transport mechanisms represented in the model is given by  
\begin{equation}
F_x(v) = \textrm{exp} \left(b_xg_x\frac{v-v_x}{v_T}\right) - \textrm{exp} \left((b_x-1)g_x\frac{v-v_x}{v_T}\right), \hspace{0.3cm} x \in \lrSet{N, C, K, NK} 
\label{eq:Flux}
\end{equation}
The term  
$b_x$ in \eqref{eq:Flux}  represents the transport bias across the membrane in either direction for a given ion channel or pump (b = 0.5 means transmembrane transport is bidirectional and symmetrical, i.e. no rectification, which means assymetrical ion flux \citep{herrera2018thermodynamic}). The thermal potential $v_T=kT/q$ (mV), where $k$ is Boltzmann's constant (mJ/$^o$K), $T$ is absolute temperature ($^o$K), and $q$ is elementary charge  (Coulombs). The Boltzmann constant can be thought of as a scaling factor between macroscopic (thermodynamic temperature) and microscopic (thermal energy) physics \citep{kalinin2005boltzmann}. The Nernst potential for each ion $x$ ({\natrium}, {\\calcium}, or {\kalium}) is given by: 
\begin{equation}
v_x = \frac{v_T}{z_x} \textrm{ln} \left(\frac{[x]_o}{[x]_i} \right)
\end{equation}
where $z_x$ is the ion valence and $[x]_o$ and $[x]_o$ represent concentrations outside and inside the cell, respectively. The reversal potential for the Na-K ATPase is given by $v_{NaK}=3v_{Na}-2v_{K} - v_{ATP}$ \citep{herrera2018thermodynamic}.

\begin{table}[h]
\centering
    \caption{All ion fluxes are given by a product of the form $I_x = a_x G_x F_x$ where $a_x$, $G_x$, and $F_x$ represent, respectively, the amplitude (normalized by membrane capacitance), gating, and driving force terms for the flux. The gating term for the \natrium-\kalium pump can be written as 1, which can be thought of as saturation. Note that the inactivation of \natrium-channels is also represented by $w$ \citep{rinzel1985excitation,avron1991minimal}. The proportion of non-inactivated \natrium channels is thus $1-w$. }
\begin{tabular}{l c c c c}
Mechanism & Name & Amplitude ($a$)& Gating ($G$) & Flux $F$
\\ \hline 
Transient Na current & $I_{NaT}(v,w)$ & $a_{Na}$ & $S_m(v) (1-w)$ & $F_{Na}(v)$ \\
L-type {\calcium} current& $I_{CaL}(v,c)$ & $a_{Ca}$ & $S_{n}(v)$ & $ F_{Ca}(v)$  \\ 
Delayed rectifier {\kalium} channel& $I_{DK}(v,w)$ & $a_{DK}$ & $w$ & $F_K(v)$  \\ 
SK {\calcium}-dependent {\kalium} channel& $I_{SK}(v,c)$ & $a_{SK}$& $ H_{SK}(c)$ & $ F_K(v)$  
\\ 
{\natrium}-{\kalium} pump & $I_{NaK}(v)$ & $a_{NaK}$& 1 & $F_{NaK}(v)$  
\\ \hline&&&&
\end{tabular}
    \label{tab:ionFluxes}
\end{table}


\paragraph{Gating.}The dynamics for $w$, the proportion of activated delayed-rectifier \kalium channels, are assumed to be logistic,  
\begin{equation}
\partial_t{w} = r_w w (S_w(v)-w)  R_w(v), \label{eq:dwdt}
\end{equation}
which yields better fits, and is more consistent with, the dynamics of activation in channel populations recorded in voltage-clamp experiments \citep{Covarrubiasetal1991,hodgkin1952quantitative,TsunodaSalkoff1995b}.
The parameter $r_w$ is the recovery rate for $w$ toward its voltage-dependent steady state $S_w(v)$. The function $R_w$ describes the voltage-dependence of the rate of activation of the channels. 

The auxiliary functions for voltage-dependent gating are given by
\begin{align}
S_j(v) &= \frac{\textrm{exp}\left(g_j \frac{v-v_j}{v_T} \right)}{1 + \textrm{exp}\left(g_j \frac{v-v_j}{v_T} \right)}, \hspace{0.3cm} j \in \lrSet{m,n,w} 
\\[1em]
R_j(v) &= \textrm{exp} \left(b_j g_j\frac{v-v_j}{v_T}\right) + \textrm{exp} \left((b_j-1)g_j\frac{v-v_j}{v_T}\right)
\label{eq:actRate}
\end{align}
where $g_j$ represents the steepness of the activation curve for \natrium ($m$), \calcium ($n$), or \kalium ($w$) channels; $v_j$ represents the half-activation voltage for those channels, and $b_j$ in \eqref{eq:actRate} represents the assymetry in the gating relative to voltage that biases the time constant for the gating process. 

The gating of the SK channel is not voltage-dependent. Instead, it depends on intracellular \calcium- binding, its activation is modeled using a Hill equation that depends on the intracellular concentration of \calcium, as used to fit data from channel recordings \citep{hirschberg1999gating}:
\begin{equation}
G_{SK}(c) = \frac{c^2}{c^2 + c_{SK}^2}
\end{equation}
where $c_{SK}$ represents the half-activation \calcium concentration for the \calcium-dependent \kalium-channels.

For the dynamics of intracellular \calcium we assume recovery toward a steady state $c_{\infty}$ at a rate $r_c$, with increments caused by the \calcium current $J_{Ca}$.
\begin{equation}
\partial_t{c} = r_{c}({c_{\infty} - c}) - k_{c}J_{CaL}(v,c).
\label{eq:dcdt}
\end{equation}
The term $k_c$ in equation~\eqref{eq:dcdt} is the conversion factor that accounts for the effect of \calcium flux across the membrane on the intracellular\calcium concentration. 

Spontaneous activity is simulated by replacing the term $J_{F}$ with a  time-dependent, Ornstein-Uhlenbeck (OU) process with amplitude $a_{F}(t)$ (pA). The mean is represented by  $\mu_{F}$ (pA) (drift term)
\citep{rudolph2003characterization}  given by \citep{gillespie1996mathematics,
  gillespie1996exact} 
\begin{eqnarray}
{a}_{F}(t+\delta) &=& a_{F}(t) \lrRound{1 -  \frac{\delta}{\tau_{F}}}  +
\lrSquare{ \mu_{F} \delta + \eta(t) ~\sqrt{d_{Stim} \delta}} ,
\end{eqnarray}
where $\delta$ is a small time step, $\tau_{F}$ is a relaxation time,  $\eta(t)$ is an independent
white noise process with zero-mean and unit standard deviation. The process has a variance
$\sigma_{F}^2 = d_{F} \delta /2$ (pA), which means that $d_{F}$ can be
approximated if an estimation of the variance of the current $a_{F}$ is
available \citep{rudolph2004method,destexhe2004novel}.  

\paragraph{Change of variables to obtain numerical solutions.}
To simplify the numerics, we change variables $$y = v/v_T,$$ and adjust all voltages accordingly as $$y_l = v_l/v_T.$$ The new equation for the normalized voltage is $$\partial_t y = \frac{\partial_t v}{v_T}$$.
To simplify the notation and reduce the number of operations during the numerical integration, we also reparametrize the amplitudes as $$A_l = \frac{a_l}{v_T~Cm}.$$ in units of 1/ms. The result is a new equation of the form
\begin{eqnarray}
\partial_t y &=& J_F-A_{NaKa}F_{NaKa}(y) \\
&&- \left(A_{KaD}w + A_{KaSK}\frac{^2}{c^2 + c_0^2}\right)F_{K}(y) \\
&& - A_{NaT}(1-w)m_{\infty}(y) F_{Na}(y) - A_{CaL}m_{13\infty}(y)F_{Ca}(y,c),
\end{eqnarray}
with driving force terms of the form
$$F_l(y) = 2 \sinh\left(\frac{y-y_l}{2}\right),$$ 
for  $l \in \left\{NaKa,KaD,KaSK,NaT,CaL \right\}$.
The term $J_{F}$ (1/ms) is the input current $I_F$ (pA) divided by $v_T C_m$.


\subsection{Parameters}

The ionic currents were modeled to fit as closely as possible the biophysical properties of those carried by channel variants expressed in mammalian neurons, and specifically CA1 PCs, where data are available. For example, the DK current is based on that mediated by  K$_v$2.1 channels, found to be the predominant channel underlying the delayed rectifier current in rat hippocampal neurons \cite{murakoshi1999identification}. The L-type {\calcium} current is based on that carried by Ca$_v$1.2 (class C) channels, found to be the predominant L-type channel isoform expressed in rat brain \cite{hell1993identification}. Additional details about the model current parameters can be found in Table \ref{tab:params}.

Wherever possible, model parameters were taken from studies in rodent (mice and rat) striatal neurons. If data were not available, we obtained parameters from other types of mammalian cell, or from studies of mammalian ion channels in expression systems like \textit{Xenopus} oocyte. Physical constants and other parameters which we would not expect to vary, such as the intra- and extracellular concentrations of ions or the cellular capacitance, were fixed. Biophysical properties of the ion channels, such as their half-activation voltages, were also fixed. The parameters we varied were primarily those corresponding to maximum current amplitudes, which can change acutely due to modulation or channel phosphorylation \citep{li1992functional,fadool1998modulation}, or chronically due to changes in ion channel expression that occur with age \citep{greer2018whole,herman1998up,veng2002regionally}. 

%PARAMETER TABLE
%\begin{landscape}
\begin{center}
\begin{footnotesize}
\rowcolors{2}{cyan!5}{white}
\begin{longtable}{p{4em} p{0.2\textwidth} p{4em} l p{0.5\textwidth}} %use units relative to the size of the document or relative to the character sizes (x\textwidth or em)
\caption{Constants and parameters. Note $v_{NaK}=3v_{Na}-2v_{K} - v_{ATP}$.} \\
\rowcolor{cyan!10}
\textbf{parameter} & \textbf{description} & \textbf{value} & \textbf{units} & \textbf{reference} \\
\endfirsthead
\rowcolor{cyan!10}
\textbf{parameter} & \textbf{description} & \textbf{value} & \textbf{units} & \textbf{reference} \\
\endhead
$k$ & Boltzmann's constant & 1.381e$^{-20}$ & mJ/K & physical constant \cite{hille2001ion} \\
$q$ & elementary charge & 1.602e$^{-19}$ & C & physical constant \cite{hille2001ion} \\
$T$ & absolute temperature & 273.15 + 37 & K & adjusted to mammalian body temperature of 37{\celsius} \cite{hille2001ion} \\
$C_m$ & membrane capacitance & 25.0 & pF & in range reported in rat CA1 PCs \cite{groc2002vivo} \\
$a_{NaT}$ & amplitude of \newline transient {\natrium} current & 1.5-3.5 & pA & set to produce currents of $\sim$3-7 nA as recorded in CA1 PCs from rats \citep{ketelaars2001sodium} and guinea pigs \cite{sah1988sodium} \\
$a_{CaL}$ & amplitude of \newline L-type {\calcium} current & 0.4 or 0.7 & pA & set to produce currents of $\sim$2-3 nA or $\sim$5-6 nA as recorded in young and aged CA1 PCs, respectively \citep{campbell1996aging} \\
$a_{DK}$ & amplitude of \newline delayed rectifier {\kalium}  current & 20-50 & pA & set to produce currents of $\sim$7-10 nA as recorded from HEK cells expressing rat Kv2.1 and $J_K$ in hippocampal neurons \citep{mohapatra2009regulation} \\
$a_{SK}$ & amplitude of {\calcium}-dependent {\kalium} current & 1.1-2.5 & pA & set to produce currents of $\sim$300 pA to 1.2 nA, depending on {\calcium} concentration, as recorded in SK-transfected cells \cite{scuvee2004electrophysiological} \\
$a_{NaK}$ & amplitude of {\natrium}/{\kalium} ATPase current & 0.015-0.035 & pA & set to produce currents of $\sim$90-250 pA, similar to but on high end of range recorded in hippocampal PCs \cite{richards2007differential} \\
$v_{Na}$ & Nernst potential for {\natrium} & 65 & mV & in range reported for mammalian cells \cite{johnston1995foundations} \\
$v_{Ca}$ & Nernst potential for {\calcium} & variable; baseline 135 & mV & in range reported for mammalian cells \cite{johnston1995foundations} \\
$v_{K}$ & Nernst potential for {\kalium} & -89 & mV & in range reported for mammalian cells \cite{johnston1995foundations} \\
$v_{ATP}$ & Nernst potential for ATP & -450 & mV &  value used in model of mammalian heart cells and based on fit to data \cite{endresen2000theory} \\
$v_{NaK}$ & Nernst potential for {\natrium}/{\kalium} ATPase & -76 & mV & calculated based on the Nernst potentials for ATP, {\natrium}, and {\kalium}, and a 3:2 stoichiometry, respectively \citep{weer1988voltage} \\
$r_{w}$ & rate of activation of delayed rectifier {\kalium} current & 1.0-2.5 & ms & in range recorded from CA1 PCs in slice \citep{martina1998functional} or hippocampal neurons in culture \cite{muller2002differential} \\
$s_{m}$ & symmetry of time constant of transient {\natrium} current & 0.5 & - & based on voltage dependence of time constant recorded in rat CA1 PCs \citep{martina1997functional} \\
$s_{n}$ & symmetry of time constant of L-type {\calcium} current & 0.5 & - & based on voltage dependence of time constant recorded in guinea pig CA1 PCs \cite{kay1987calcium} \\
$s_{w}$ & symmetry of time constant of delayed rectifier {\kalium} current & 0.3 & - & based on fit; if higher (0.5-0.7) APs are the wrong shape and do not ride on sufficient plateau potential compared to recordings \\
$v_{m}$ & half-activation potential of {\natrium} current & -19 & mV & in range reported for transient {\natrium} channels in CA1 PCs  \citep{estacion2010sodium,gasparini2002phosphorylation,martina1997functional} \\
$v_{w}$ & half-activation potential of delayed rectifier {\kalium} current & -1 & mV & in range reported for rat Kv2.1 channels expressed in COS-1 cells \cite{murakoshi1999identification} \\
$v_{n}$ & half-activation potential of L-type {\calcium} current & 3 & mV & in range recorded for high-voltage activated {\calcium} currents in rat CA1 PCs \cite{magee1995characterization}; see also recordings from oocytes \citep{xu2001neuronal} or HEK cells \citep{balasubramanian2009optimization} expressing Ca$_v$1.2 channels \\
$c_{SK}$ & half-activation {\calcium} concentration for SK current & 740 & nM & based on recordings from oocytes expressing rat SK channel variant \citep{hirschberg1998gating} \\
$z_{m}$ & activation slope of transient {\natrium} current & 4.0 & - &  \\
$g_{n}$ & activation slope of L-type {\calcium} current & 4.0 & - & \\
$g_{w}$ & activation slope of delayed rectifier {\kalium} current & 4.0 & - & fit to data from rat brain delayed rectifier channels \cite{vandongen1990alteration} \\
%$z_{SK}$ & activation slope factor of {\calcium}-dependent {\kalium} current & 2.0 & - & based on recordings from oocytes expressing rat SK channel variant \citep{hirschberg1998gating} \\
$c_{\infty}$ & minimum intracellular {\calcium} concentration & 100 & nM & approximate resting intracellular {\calcium} concentration reported in rat CA1 PCs \citep{oh2013altered,magee1996dihydropyridine,gant2006early} \\
$r_{c}$ & {\calcium} removal rate constant &  8e$^{-3}$ to 1e$^{-3}$ & - & adjusted to produce {\calcium} dynamics as recorded in rat CA1 PCs \citep{oh2013altered} \\
$k_{c}$ & {\calcium} current to concentration conversion factor & 8e$^{-6}$ to 6e$^{-6} $ & - & adjusted to produce {\calcium} dynamics as recorded in rat CA1 PCs \citep{oh2013altered} \\
\label{tab:params}
\end{longtable}
\end{footnotesize}
\end{center}
%\end{landscape}

% COMPARISON PARAMETER TABLE
\begin{table}[h!]
\centering
 \caption{Parameters used to produce different firing patterns in the yPC model. The normalization of amplitudes was calculated with $v_T C_m=668.171$ mv~pF. Amplitudes in pA are included (in parentheses) for reference with respect to experimental measures.The amplitude for the L-type \calcium current in the aPC was set to $a_{CaL}$=467.719~ pA, which is equivalent to $A_{CaL}$=0.7~(1/ms)} 
\begin{footnotesize}
\rowcolors{2}{white}{cyan!5}
\begin{tabular}{p{5em} p{5em}  p{5em}  p{5em}} % spaces between the widths are important
\textbf{parameter} & \textbf{adaptive \newline firing} & \textbf{conditional bursting} & \textbf{spontaneous bursting} \\
$A_{NaK}$ ($a_{NaK}$ pA)& 0.015 (10.0226) & 0.020 (13.3634)& 0.040 (26.7268)\\
$A_{KD}$ ($a_{KD}$ pA)& 40.0 (26726.8)& 20.0 (13363.4)& 30.0 (20045.1)\\
$A_{SK}$ ($a_{SK}$ pA)& 1.1 (734.988)& 2.5 (1670.43) & 1.1 (734.988)\\
$A_{Na}$ ($a_{Na}$ pA) & 1.5 (1002.26) & 2.0 (1336.34)& 4.0 (2672.68)\\
$A_{CaL}$ ($a_{CaL}$ pA)& 0.4 (267.268) & 0.4 (267.268)& 0.4 (267.268)\\
$r_{KD}$ & 1.0 & 2.5 & 1.0 \\
$r_{Ca}$ & 1e$^{-3}$ & 5e$^{-3}$ & 1e$^{-2}$ \\
$k_{Ca}$ & 8e$^{-6}$ & 6e$^{-6}$ & 6e$^{-6}$ \\
\end{tabular}
\label{tab:regimes}
\end{footnotesize}
\end{table}


\bibliographystyle{plainnat}
\bibliography{membraneBiophysics,basalGanglia}
\end{document}
