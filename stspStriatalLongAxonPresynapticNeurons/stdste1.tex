%& -shell-escape
%\documentclass[8pt]{beamer}
%\input{mahvPreambleBeamer}
%\documentclass[10pt]{article}
\documentclass[11pt,landscape]{article}
\usepackage[utf8]{inputenc}
/Users/curandero/docs/mahvPreamble.tex
\hypersetup{
  pdftitle={Herrera-Valdez et al. Simple model for short-term
    plasticity}
}
\usepackage{python}
\usepackage[left=2.5cm,right=2.5cm,top=0cm,bottom=2cm]{geometry}


%0. El balance entre la facilitación y la depresión pre-sináptica a corto plazo explica la variabilidad de respuestas postsinápticas en terminales estriato-estriatales

% 1. "Geometría y dinámica detrás de la interacción entre facilitación
% y depresión pre-sináptica en terminales estriato-estriatales"  

% 2. "Geometría y dinámica detrás de la diversidad en plasticidad sináptica de corto plazo en terminales GABAérgicas del estriado"

% 3. "Análisis geométrico y simulación de la interacción entre facilitación y depresión presináptica en el núcleo estriado"

% 4. "Análisis geométrico y simulación de la interacción entre facilitación y depresión presináptica en terminales estriato-estriatales"


\begin{document}


\begin{flushleft}
\begin{Large}
The balance between pre-synaptic short-term facilitation and
depression explains the variability of observable responses in striato-striatal
GABAergic terminals.
\end{Large}
\\
% \begin{large}
% Marco Arieli Herrera Valdez $^{1}$, 
% Janet Barroso $^{2}$, 
% José Bargas $^{2}$
% \end{large}
% \\
% $^{1}$ \textit{Instituto de Matemáticas, Universidad Nacional Autónoma
% de México, México, D.F. 04510, México}
% \\
% $^{2}$ \textit{Instituto de Fisiología Celular, Universidad Nacional Autónoma
% de México, México, D.F. 04510, México}
\end{flushleft}



% -----------------------------------------------
\section{Simple model of short-term plasticity}


Assume that a presynaptic cell fires action potentials at times $t_k$,
$k =0,1,...,N$, with interspike intervals $y_{k} = t_{k}-t_{k-1}$, $k=1,...,N$. 
Let $x$ be the occupancy of the readily releasable pool and $p$ the
probability of release. The presynaptic release
can then be described by two variables with dynamics given by
\begin{eqnarray}
\partial_t x &=& \frac{x\lrRound{1-x}}{\tau_{r}} -  p x \sum_{i=1}^{N} \phi\lrRound{t-t_i} ,
\label{eq:occupancy}
\\
\partial_t p &=&\frac{p_{\infty}-p}{\tau_{f}} + \lrRound{1-p}  \sum_{i=1}^{N}
h_i \phi \lrRound{t-t_i},
\label{eq:prelease}
\end{eqnarray}
where $x_{\infty}$, $p_{\infty}$, $\tau_{r}$ and $\tau_{f}$, represent
the steady state occupancy, the steady state probability of release,
and the recovery and facilitation time constants, respectively. Note
the equation for occupancy has logistic dynamics between pulses with
two limiting values 0 and $x_{\infty}$, which are, respectively, repelling and
attractor states for $x$.   
The
occupancy and probability of release can be thought of as the
depression and facilitation variables, respectively. The function
$\phi$ is a small pulse triggered by presynaptic action potentials
that enables changes in both, the occupancy and the probability of
release.  The increase in $p$ at a spike time $t_i$ is a fraction
$h_i$ of $1-p$ and 
could, a priori, depend on the interval of time between the current,
and the last presynaptic action
potential \citep[see][for a review]{hennig2013Theoretical}.

The steady states $x_{\infty}$ and $p_{\infty}$ could increase or decrease to
model long-term enhancement or depression, respectively. Note that
setting the steady state value of occupancy to one, is equivalent to
assume that long term enhancement could be due only to an increase in
the steady state probability of release.  Note then that the dynamics
of $x$ capture depression by depletion of the readily releaseable pool
of vesicles, whereas the dynamics of $p$ capture facilitation by
accumulation of calcium or some other factor that increases the
probability of vesicle release. 


\subsection{Enhancement dynamics}

Between action potentials,
the dynamics of $p$ are given by
\begin{equation}
  p(t) = p_{\infty} - \lrRound{p_{\infty} - p_0} \exp \lrRound{-t/ \tau_p}
\end{equation} 
where $p_0$ is the initial condition.  Assume that $p(0)=p_{\infty}$ and
that presynaptic actions potentials occur at a times
$0<t_0<t_1<...<t_n$, after a long time without firing. The probability
of release does not change after the first action potential. At each
action potential after $t_0$, 
\begin{displaymath}
p \mapsto p+ h_1 (1-p)  = h_1 +
(1-h_1) p.
\end{displaymath}
Then,  
\begin{equation}
p(t_1)= h_1 + (1-h_1) p_{\infty}
\end{equation}
which represents a new initial contidion for $p$. For $t \in 
(t_1,t_2)$, the dynamics of $p$ are then given by
\begin{eqnarray}
p(t) 
&=&
p_{\infty} - \lrRound{p_{\infty} - p(t_1) }\exp
\lrRound{-\frac{t-t_1}{\tau_p}},
\\ 
&=& p_{\infty} - \lrSquare{p_{\infty} - p_{\infty} - h_1   (1-p_{\infty})}
\exp \lrRound{-\frac{t-t_1}{\tau_p}} ,
\\
&=& p_{\infty} + h_1   (1-p_{\infty})
\exp \lrRound{-\frac{t-t_1}{\tau_p}} .
\end{eqnarray}
At $t=t_2$, $p$ jumps to $p + (1-p)h_2 = h_2+
(1-h_2)p$, so that
\begin{eqnarray}
p(t_2)
&=&
h_2 
+ (1-h_2) \lrSquare{p_{\infty} + h_1 (1-p_{\infty}) 
\exp \lrRound{-\frac{t_2-t_1}{\tau_p}}}.
\end{eqnarray}
The resulting dynamics for $t \in (t_2,t_3)$ are then
\begin{eqnarray}
p(t) 
&=& 
p_{\infty} - \lrRound{p_{\infty} - p(t_2) }\exp
\lrRound{-\frac{t-t_2}{\tau_p}},
\nonumber \\
&=& 
p_{\infty} - \lrSet{p_{\infty}   
\textcolor{blue}{-
  h_2 - (1-h_2) \lrSquare{
    p_{\infty} + h_1 (1-p_{\infty}) 
    \exp \lrRound{-\frac{t_2-t_1}{\tau_p}}
}}}
\exp \lrRound{-\frac{t-t_2}{\tau_p}},
\nonumber \\
&=& 
p_{\infty} - \lrSet{p_{\infty} -  
  h_2 - (1-h_2) p_{\infty}
 - h_1  (1-h_2) (1-p_{\infty}) 
    \exp \lrRound{-\frac{t_2-t_1}{\tau_p}}}
\exp \lrRound{-\frac{t-t_2}{\tau_p}},
\nonumber \\
&=& 
p_{\infty} - \lrSquare{
-    h_2  
+
p_{\infty} \lrRound{1 - (1-h_2) }
- h_1 (1-h_2) (1-p_{\infty}) 
    \exp \lrRound{-\frac{t_2-t_1}{\tau_p}}}
\exp \lrRound{-\frac{t-t_2}{\tau_p}},
\nonumber \\
&=& 
p_{\infty} + \lrSquare{
    h_2  
- p_{\infty}  h_2 
+ h_1 (1-h_2) (1-p_{\infty}) 
    \exp \lrRound{-\frac{t_2-t_1}{\tau_p}}}
\exp \lrRound{-\frac{t-t_2}{\tau_p}},
\nonumber \\
&=& 
p_{\infty} + \lrSquare{
    h_2  (1- p_{\infty})
+ h_1 (1-h_2) (1-p_{\infty}) 
    \exp \lrRound{-\frac{t_2-t_1}{\tau_p}}}
\exp \lrRound{-\frac{t-t_2}{\tau_p}},
\nonumber \\
&=& 
p_{\infty} + (1- p_{\infty}) \lrSquare{
    h_2  
+ h_1 (1-h_2)  
    \exp \lrRound{-\frac{t_2-t_1}{\tau_p}}}
\exp \lrRound{-\frac{t-t_2}{\tau_p}},
\nonumber \\
&=& 
p_{\infty} + (1- p_{\infty}) \lrSquare{
    h_2  \exp \lrRound{-\frac{t-t_2}{\tau_p}}
+ h_1 (1-h_2)  
    \exp \lrRound{-\frac{t-t_1}{\tau_p}}}.
\end{eqnarray}
\newpage
Once again, for the action potential at time $t=t_3$,
\begin{eqnarray}
p(t_3) &=& h_3 + (1-h_3)
\lrSet{p_{\infty} + (1-p_{\infty})  \lrSquare{ 
h_2 \exp \lrRound{-\frac{t_3-t_2}{\tau_p}}
 + h_1 (1-h_2) 
    \exp \lrRound{-\frac{t_3-t_1}{\tau_p}} }}
\end{eqnarray}
and for $t \in (t_3,t_4)$, 
\begin{eqnarray}
p(t) 
&=& 
p_{\infty} - \lrRound{p_{\infty} - p(t_3) }
\exp \lrRound{-\frac{t-t_3}{\tau_p}},
\nonumber \\
&=& 
p_{\infty} - \lrRound{p_{\infty} -
\textcolor{blue}{
h_3 - (1-h_3)
\lrSet{p_{\infty} + (1-p_{\infty})  \lrSquare{ 
h_2 \exp \lrRound{-\frac{t_3-t_2}{\tau_p}}
 + h_1 (1-h_2) 
    \exp \lrRound{-\frac{t_3-t_1}{\tau_p}} }}
}}
\exp\lrRound{-\frac{t-t_3}{\tau_p}},
\nonumber \\
&=& 
p_{\infty} - \lrRound{p_{\infty} 
\textcolor{blue}{
- h_3 - (1-h_3)p_{\infty} 
- (1-h_3)(1-p_{\infty})  \lrSquare{ 
h_2 \exp \lrRound{-\frac{t_3-t_2}{\tau_p}}
+ h_1 (1-h_2) 
\exp \lrRound{-\frac{t_3-t_1}{\tau_p}} }}
}
\exp\lrRound{-\frac{t-t_3}{\tau_p}},
\nonumber \\
&=& 
p_{\infty} - \lrRound{p_{\infty} h_3
\textcolor{blue}{
-h_3  - (1-h_3)(1-p_{\infty})  \lrSquare{ 
h_2 \exp \lrRound{-\frac{t_3-t_2}{\tau_p}}
 + h_1 (1-h_2) 
    \exp \lrRound{-\frac{t_3-t_1}{\tau_p}} }}
}
\exp\lrRound{-\frac{t-t_3}{\tau_p}},
\nonumber \\
&=& 
p_{\infty} + \lrRound{
  h_3 \lrRound{1-p_{\infty}} 
  \textcolor{blue}{
    + (1-h_3)(1-p_{\infty})  \lrSquare{ 
      h_2 \exp \lrRound{-\frac{t_3-t_2}{\tau_p}}
      + h_1 (1-h_2) 
      \exp \lrRound{-\frac{t_3-t_1}{\tau_p}} }}
}
\exp\lrRound{-\frac{t-t_3}{\tau_p}},
% \nonumber \\
% &=& 
% p_{\infty} - (1-p_{\infty}) \lrRound{
% \textcolor{blue}{
%   h_3 \exp\lrRound{-\frac{t-t_3}{\tau_p}}
%   + h_2 (1-h_3)  
%   \exp \lrRound{-\frac{t_3-t_2}{\tau_p}} \exp\lrRound{-\frac{t-t_3}{\tau_p}}
%   + h_1 (1-h_2) (1-h_3)  
%   \exp \lrRound{-\frac{t_3-t_1}{\tau_p}} }
% \exp\lrRound{-\frac{t-t_3}{\tau_p}},
% }
\nonumber \\
&=& 
p_{\infty} + (1-p_{\infty}) \lrSquare{
\textcolor{blue}{
h_3 \exp\lrRound{-\frac{t-t_3}{\tau_p}}
+ h_2 (1-h_3)  
 \exp \lrRound{-\frac{t-t_2}{\tau_p}} 
 + h_1 (1-h_2) (1-h_3)   
    \exp \lrRound{-\frac{t-t_1}{\tau_p}} }
}.
\end{eqnarray}

In general, for $t \in (t_{m},t_{m+1})$
\begin{eqnarray}
p(t)
&=& 
p_{\infty} + (1-p_{\infty}) 
\sum_{k=1}^{m} 
h_k \lrSquare{
\prod_{l=k+1}^{m}  
\lrRound{1- h_{l}}} 
\exp \lrRound{-\frac{t-t_k}{\tau_p}}. 
\label{eq:pBetweenAP}
\end{eqnarray}
At $t=t_{m+1}$, 
\begin{eqnarray}
p(t_{m+1})
&=&h_{m+1} + \lrRound{1-h_{m+1}}
\lrSet{
 p_{\infty} + \lrRound{1-p_{\infty}} 
\sum_{k=1}^{m} 
h_k \lrSquare{
\prod_{l=k+1}^{m}  
\lrRound{1- h_{l}}} 
\exp \lrRound{-\frac{t_{m+1}-t_k}{\tau_p}}
}. 
\label{eq:pAP}
\end{eqnarray}

\paragraph{Simplification.} 
Assume that $h_k =h$ for
$k=1,...,n$.  Then, between action potentials occuring at times $t_m$
and $t_{m+1}$,  equation \eqref{eq:pBetweenAP} simplifies to
\begin{eqnarray}
p(t)
&=& 
p_{\infty} + (1-p_{\infty}) 
\sum_{k=1}^{m} 
h \lrSquare{
\prod_{l=k+1}^{m}  
\lrRound{1- h}} 
\exp \lrRound{-\frac{t-t_k}{\tau_p}}. 
\nonumber \\ 
&=& 
p_{\infty} + (1-p_{\infty}) 
h \sum_{k=1}^{m} 
\lrRound{1- h}^{m-k} 
\exp \lrRound{-\frac{t-t_k}{\tau_p}}. 
\end{eqnarray}
At $t=t_{m+1}$, the probability of release is
\begin{eqnarray}
p(t_{m+1})
&=&
p_{\infty} + (1-p_{\infty}) 
h \sum_{k=1}^{m} 
\lrRound{1- h}^{m-k} 
\exp \lrRound{-\frac{t_{m+1}-t_k}{\tau_p}}. 
\end{eqnarray}
If the interspike intervals are such that $\delta = t_{m} - t_{m-1}$
for all $m \in \lrSet{1,...,n}$, then  equation \eqref{eq:pBetweenAP}
simplifies further to
\begin{eqnarray}
p(t)
&=& 
p_{\infty} + (1-p_{\infty}) 
h \exp \lrRound{-\frac{t-t_{m}}{\tau_p}}
 \sum_{k=1}^{m} 
\lrRound{1- h}^{m-k} 
\exp \lrRound{-(m-k) \frac{\delta}{\tau_p}}
\nonumber \\
&=& 
p_{\infty} + (1-p_{\infty}) 
 \exp \lrRound{-\frac{t-t_{m}}{\tau_p}}
h \lrRound{1- h}^{m}
\exp \lrRound{- \frac{m \delta}{\tau_p}}
 \sum_{k=1}^{m} 
 \lrRound{1- h}^{-k} 
\exp \lrRound{\frac{k \delta}{\tau_p}}
\nonumber \\
&=& 
p_{\infty} + (1-p_{\infty}) 
 \exp \lrRound{-\frac{t-t_{m}}{\tau_p}}
h \lrRound{1- h}^{m}
\exp \lrRound{- \frac{m \delta}{\tau_p}}
 \sum_{k=1}^{m} \lrRound{
 \frac{\exp \lrRound{ \delta/ \tau_p}}
{1- h}}^{k}
\nonumber \\
&=& 
p_{\infty} + (1-p_{\infty}) 
 \exp \lrRound{-\frac{t-t_{m}}{\tau_p}}
h \lrRound{1- h}^{m}
\exp \lrRound{- \frac{m \delta}{\tau_p}}
\frac
{1 - \frac{\exp \lrRound{ \lrRound{m+1}\delta /\tau_p}}
{\lrRound{1- h}^{m+1}}}
{1- \frac{\exp \lrRound{ \delta/\tau_p}}
{1- h}}
\nonumber \\
&=& 
p_{\infty} + (1-p_{\infty}) 
 \exp \lrRound{-\frac{t-t_{m}}{\tau_p}}
h 
\frac
{ \lrRound{1- h}^{m}
\exp \lrRound{- m \delta / \tau_p}
- \frac
{\exp \lrRound{  \delta / \tau_p}}
{\lrRound{1- h}}
}
{1- \frac
{\exp \lrRound{ \delta / \tau_p}}
{1- h}
}
\nonumber \\
&=& 
p_{\infty} + (1-p_{\infty}) 
 \exp \lrRound{-\frac{t-t_{m}}{\tau_p}}
h 
\frac
{ \lrRound{1- h}^{m+1}
\exp \lrRound{- m \delta / \tau_p}
- {\exp \lrRound{  \delta / \tau_p}}
}
{\lrRound{1- h} - \exp \lrRound{ \delta / \tau_p}}
\nonumber \\
&=& 
p_{\infty} + (1-p_{\infty}) 
 \exp \lrRound{-\frac{t-t_{m}}{\tau_p}}
F(h,\delta,m)
\end{eqnarray}
where 
\begin{eqnarray}
F(h,\delta, m) 
&=& 
h 
\frac
{ \lrRound{1- h}^{m+1}
\exp \lrRound{- m \delta / \tau_p}
- {\exp \lrRound{  \delta / \tau_p}}
}
{\lrRound{1- h} - \exp \lrRound{ \delta / \tau_p}}
\end{eqnarray}
At $t=t_{m+1}$, 
\begin{eqnarray}
p(t_{m+1})
&=&
\end{eqnarray}

\subsubsection{Frequency-dependent enhancement}
Consider first Eqn.~\eqref{eq:prelease} with frequency-dependent
increase in the probability of release given by 
\begin{equation}
h_i =g(h_i) =\frac{h_0}{h_0 + h_i}, 
\label{eq:epsilon}
\end{equation}
where $h_0$ is the interspike interval at which the increase in the probability
of release is half of its current value. 


\subsection{Analysis of occupancy (depression)}

For simplicity in the analysis, let $\phi$ be an instantaneous pulse
for each presynaptic spike. Also, assume the probability of release
and the facilitation factor $h$ are constant and an initial state of
occupancy $x(0)=x_{\infty}$. 
Under those conditions, the occupancy of readily
releasable sites after the the $n$th pulse is given 
\begin{eqnarray}
x_n
&=& 
\frac{
x_{\infty} \lrRound{1-p}^{n+1} 
}
{
\lrRound{1-p}^{n} 
+ p
\sum_{k=1}^{n} \lrRound{1-p}^{n-k} \exp\lrSquare{- (t_n-t_{n-k}) r_x }
}
\label{tnOccup1}
\\
&=& 
\frac{
x_{\infty} \lrRound{1-p}
}{
1
+ p
\sum_{k=1}^{n} \frac{ \exp\lrSquare{- (t_n-t_{n-k}) r_x }}{\lrRound{1-p}^{k}}
}
\label{tnOccup2}
% \\
% &=& 
% \frac{
% x_{\infty} \lrRound{1-p}
% }{
% 1
% + p \exp\lrSquare{- t_n r_x}
% \sum_{i=1}^{n} \lrRound{1-p}^{-i} \exp\lrSquare{t_{n-i} r_x }
% }
% \label{tnOccup3}
\end{eqnarray}

\paragraph{Triggering pulses.}
A more realistic approach would be to define $\phi$
so that its time course resembles that of calcium activation, during
which neurotransmitter is released, the occupancy  of vesicle docking sites drops, and
the probability of release changes. 
One such function could be
\begin{eqnarray}
\phi(y) &=& H(y)  \frac{y}{\tau_{\alpha}} \exp \lrRound{-\frac{y}{\tau_{\alpha}}}
\label{eq:alpha}
\end{eqnarray}
where $H(y)$ is the heavy-side function, taking the value 0 if for
$y<0$, and 1 otherwise. 

\paragraph{Enhancement as a function of interspike intervals.}
Facilitation (\textit{i.e.} increase in $p$), can be assumed
to depend on the presynaptic interspike intervals $h_k=t_{k} -t_{k-1}$ for $k=2,3,...$. 
For instance, 
\begin{equation}
q(h)= \bar{q} \frac{h}{\tau_q}   e^{(h-h_{*})/\tau_q}
\end{equation}
where $\tau_q$ can be regarded as a facilitation time constant, $h_*$
as an optimal interspike interval for facilitation, and
$\bar{q}$ is the maximum increase in the probability of release. 
 





\newpage
\section{Mathematical details}


Let $(x_k,p_k) = (x(t_k),p(t_k))$ represent the states of the system
at the time of arrival of the $k$th pulse.
From \eqref{eq:occupancy}-\eqref{eq:prelease}, the dynamics of $(x,p)$
for $t \in (t_i, t_{i+1})$ satisfy 
\begin{eqnarray}
x(t) 
&=& \frac{x_{\infty} {x_i}}{{x_i} + \lrRound{x_{\infty}- x_i}
\exp\lrRound{- (t-t_i) x_{\infty} / \tau_x }},
\label{xLogisticSolution}
\\
p(t) 
&=& p_{\infty} - \lrRound{p_{\infty}- p_i}
   \exp\lrRound{-\frac{(t-t_i)}{\tau_p}}. 
\label{pLinearSolution}
\end{eqnarray}

At $t=t_{k+1}$, the time of the $(k+1)$th 
spike,  the state $(x,p)(t)$ jumps to the value 
\begin{eqnarray}
x_{k+1} 
&=& \frac{x_{\infty} {x_{k}}  \lrRound{1-p_{k+1}} }{{x_{k}} + \lrRound{x_{\infty}- x_{k}}
\exp\lrRound{- (t_{k+1}-t_{k}) x_{\infty} / \tau_x }}
\\
p_{k+1} 
&=& \lrSquare{{p_{k}} + \lrRound{p_{\infty}- p_{k}}
\exp\lrRound{- (t_{k+1}-t_{k})  / \tau_p }} \lrRound{1-p_{k+1} x_{k+1}}
\end{eqnarray}

\subsection{Analysis of occupancy  (depression) in response to
  $\delta$-pulses for constant $p$}

At $t=t_{i+1}$, the time of the $(i+1)$th 
pulse,  the occupancy jumps to the value 
\begin{eqnarray}
x_{i+1} 
&=& \frac{x_{\infty} {x_{i}}  \lrRound{1-p} }{{x_{i}} + \lrRound{x_{\infty}- x_{i}}
\exp\lrRound{- (t_{i+1}-t_{i}) x_{\infty} / \tau_x }}
% \\
%&=& \frac{x_{\infty}}{{1} + \lrRound{ \frac{x_{\infty}}{x_{k-1}} -1}
%\exp\lrRound{- t_{k} x_{\infty} / \tau_x }} \lrRound{1-p}
\label{tkOccup}
\end{eqnarray}
for $i \in \lrSet{0,...,N-1}$. 


The general expression in \eqref{tkOccup} can be used recursively to
investigate the behavior of the occupancy after a series of pulses. To
do so, let $i=1$, $r_x=x_{\infty} / \tau_x$, and replace the occupancy
$x_0=x_{\infty}(1-p)$ after the first pulse  into Eqn.~\eqref{tkOccup}.
%  and
% $\delta_{k}=t_{k}-t_{k-1}$,  represent the interspike interval between the
% $(k-1)$th and the $k$th pulse, for $k=2,...,N$. 
Then, the occupancy at $t=t_1$ is
\begin{eqnarray}
x_1 
&=& \frac{x_{\infty} \lrRound{1-p}}
{1 + \lrSquare{ \frac{x_{\infty}}{\textcolor{blue}{x_0}} -1}
\exp\lrSquare{- (t_1-t_0) r_x }}
\nonumber
\label{t1Occup0}
\\
&=& \frac{x_{\infty} \lrRound{1-p}}
{1 + \lrSquare{ \frac{x_{\infty}}
{\textcolor{blue}{x_{\infty}(1-p) }} -1}
\exp\lrSquare{- (t_1-t_0) r_x }}
\nonumber
\label{t1Occup1}
\\
&=& \frac{ x_{\infty} \lrRound{1-p}}
{1 + \lrRound{ 
\textcolor{blue}{\frac{p}{1-p}}}
\exp\lrSquare{- (t_1-t_0) r_x }}
\nonumber
\label{t1Occup2}
\\
&=& \frac{ x_{\infty} \lrRound{1-p}^2}
{ \lrRound{1-p} + p \exp\lrSquare{- (t_1-t_0) r_x }}
\label{t1Occup3}
\end{eqnarray}

The occupancy at $t=t_2$ is
\begin{eqnarray}
x_2 
&=& \frac{x_{\infty} \lrRound{1-p}}
{1 + \lrSquare{ \frac{x_{\infty}}{\textcolor{blue}{x_1}} -1}
\exp\lrSquare{- (t_2-t_1) r_x }}
\nonumber
\label{t2Occup0}
\\
&=& \frac{x_{\infty} \lrRound{1-p}}
{1 + \lrSquare{ \frac{x_{\infty}}
    {\textcolor{blue}{
        \frac{ x_{\infty} \lrRound{1-p}^2}
        { \lrRound{1-p} + p \exp\lrSquare{- (t_1-t_0) r_x }}
      }} -1}
\exp\lrSquare{- (t_2-t_1) r_x }}
\nonumber
\label{t2Occup1}
\\
&=& \frac{x_{\infty} \lrRound{1-p}}
{1 + \lrSquare{ 
    \textcolor{blue}{
      \frac
      { \lrRound{1-p} + p \exp\lrSquare{- (t_1-t_0) r_x }}
      { \lrRound{1-p}^2}
    } 
    -1
  }
  \exp\lrSquare{- (t_2-t_1) r_x }}
\nonumber
\label{t2Occup2}
\\
&=& \frac{x_{\infty} \lrRound{1-p}^3}
{\lrRound{1-p}^2 + \lrSquare{ 
    \textcolor{blue}{
      { \lrRound{1-p} 
        + p \exp\lrSquare{- (t_1-t_0) r_x }}
    } 
    - \lrRound{1-p}^2
  }
  \exp\lrSquare{- (t_2-t_1) r_x }
}
\nonumber
\label{t2Occup3}
\\
&=& \frac{x_{\infty} \lrRound{1-p}^3}
{
  \lrRound{1-p}^2 + \lrSquare{ 
    \textcolor{blue}{  
      \lrRound{ \lrRound{1-p}     - \lrRound{1-p}^2}
      \exp\lrSquare{- (t_2-t_1) r_x }
      + p \exp\lrSquare{- (t_2-t_0) r_x }}
  } 
}
\nonumber
\label{t2Occup4}
\\
&=& \frac{x_{\infty} \lrRound{1-p}^3}
{
  \lrRound{1-p}^2 + 
  p\lrRound{1-p}
  \exp\lrSquare{- (t_2-t_1) r_x }
  + p \exp\lrSquare{- (t_2-t_0) r_x }
},
\label{t2Occup5}
\end{eqnarray}
from which it is clear that a pattern starts to emerge.
The occupancy at $t=t_3$ is then 
\begin{eqnarray}
x_3 
&=& \frac{x_{\infty} \lrRound{1-p}}
{1 + \lrSquare{ \frac{x_{\infty}}{\textcolor{blue}{x_2}} -1}
\exp\lrSquare{- (t_3-t_2) r_x }}
\nonumber
\label{t2Occup0}
\\
&=& \frac{x_{\infty} \lrRound{1-p}}
{1 + \lrSquare{ \frac{x_{\infty}}
    {\textcolor{blue}{
        \frac{x_{\infty} \lrRound{1-p}^3}
        {
          \lrRound{1-p}^2 + 
          p\lrRound{1-p}
          \exp\lrSquare{- (t_2-t_1) r_x }
          + p \exp\lrSquare{- (t_2-t_0) r_x }
        }
      }
    }
    -1}
\exp\lrSquare{- (t_3-t_2) r_x }}
\nonumber
\label{t2Occup1}
\\
&=& \frac{ x_{\infty} \lrRound{1-p}}
{1 + \lrSquare{ 
    {\textcolor{blue}{
        \frac
        {
          \lrRound{1-p}^2 + 
          p\lrRound{1-p}
          \exp\lrSquare{- (t_2-t_1) r_x }
          + p \exp\lrSquare{- (t_2-t_0) r_x }
        }{ \lrRound{1-p}^3}
      }
    }
    -1}
\exp\lrSquare{- (t_3-t_2) r_x }}
\nonumber
\label{t2Occup2}
\\
&=& \frac{ x_{\infty} \lrRound{1-p}^4 }
{\lrRound{1-p}^3 + \lrSquare{ 
    {\textcolor{blue}{
        {
          \lrRound{1-p}^2 + 
          p\lrRound{1-p}
          \exp\lrSquare{- (t_2-t_1) r_x }
          + p \exp\lrSquare{- (t_2-t_0) r_x }
        }
      }
    }
    - \lrRound{1-p}^3}
\exp\lrSquare{- (t_3-t_2) r_x }}
\nonumber
\label{t2Occup3}
\\
&=& \frac{ x_{\infty} \lrRound{1-p}^4 }
{\lrRound{1-p}^3 + 
  {\textcolor{blue}{
      p \lrRound{1-p}^2 \exp\lrSquare{- (t_3-t_2) r_x } 
      + 
      p\lrRound{1-p}
      \exp\lrSquare{- (t_3-t_1) r_x }
      + p \exp\lrSquare{- (t_3-t_0) r_x }
    }
  }
}
\label{t2Occup4}
\end{eqnarray}

In general, for the $n$th pulse,
\begin{eqnarray}
x_n
&=& 
\frac{
x_{\infty} \lrRound{1-p}^{n+1} 
}
{
\lrRound{1-p}^{n} 
+ p
\sum_{i=1}^{n} \lrRound{1-p}^{n-i} \exp\lrSquare{- (t_n-t_{n-i}) r_x }
}
\label{tnOccup1}
\\
&=& 
\frac{
x_{\infty} \lrRound{1-p}
}{
1
+ p
\sum_{i=1}^{n} \lrRound{1-p}^{-i} \exp\lrSquare{- (t_n-t_{n-i}) r_x }
}
\label{tnOccup2}
\\
&=& 
\frac{
x_{\infty} \lrRound{1-p}
}{
1
+ p \exp\lrSquare{- t_n r_x}
\sum_{i=1}^{n} \lrRound{1-p}^{-i} \exp\lrSquare{t_{n-i} r_x }
}
\label{tnOccup3}
\end{eqnarray}



\newpage
\section{Abstract from Barroso, Herrera-Valdez, and Bargas}
The neostriatal neuronal population is composed mainly by spiny projection
neurons (SPNs), which make synapses with each other via their local
axon collaterals (approx. 100mm long) shaping the activity of a localized feedback
inhibitory circuit. The other 5-10\% of the neuronal population is
composed by different classes of interneurons, mainly
GABAergic. The synapses between GABAergic interneurons and SPNs form a
feedforward inhibitory circuit in the striatum by connecting hundreds
of SPNs, which may be as far as 1 mm. Both inhibitory circuits are richly inervated
by dopaminergic neurons from the substantia nigra pars compacta (SNc),
and therefore both circuits are affected by dopamine (DA) loss, like
in Parkinson disease. However, little is known of how the dopamine
loss affects each synapse form the feedforward inhibitory circuit. To
asses this question, we favor the feedforward circuit by using
intraestriatal field stimulation 1 mm away from the SPN recorded. We
show that using field stimulation it is possible to distinguish
between synapses of the feedforward circuit characterized by their
short term synaptic plasticity (STSP); finding that there are at least
three different types of inhibition in the feedforward circuit:
depressing synapses (putative fast spiking (FS) to SPN synapse),
facilitating synapses (putative ???(LTS) to SPN synapse) and a novel
combination of facilitating/depressing synapses (possibly yet another kind
of LTS-SPN). In addition, we use intensity-amplitude experiments to
reveal different properties of each synapse. We then found using
cluster analysis that
each response corresponds to a different kind
of synapse. Using the 6-hydroxydopamine (6-OHDA) rodent model of
Parkinsonism we demonstrate how each of this synapses are
differentially affected, showing that depressing synapses are the less
affected by DA deprivation, whereas the inhibition form both
facilitating and facilitating/depressing synapses are enhanced
(statistical significance).  Moreover, evoked IPSCs after DA loss in
facilitating and facilitating/depressing synapse are more likely to occur than
in control conditions. Taken together, our observations  suggest that
these synapses take control of the feedforward 
inhibitory circuit in Parkinsonism and depressing synapses dominate
such a circuit in control conditions.

\bibliographystyle{plainnat}
\bibliography{membraneBiophysics}

\end{document}
