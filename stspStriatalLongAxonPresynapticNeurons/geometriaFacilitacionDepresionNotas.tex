
\textbf{Slide 1}
1. Plasticidad:  caracterización mediante el cambio en corriente
post-synáptica.
2. Biofísica y bioquímica asociadas con espacios de parámetros
difíciles de caracterizar, experimentalmente, o indirectamente
optimizando con técnicas matemáticas.

\textbf{Slide 2}
1. Facilitación: Cambios en la probabilidad de liberación
2. Depresión:  Decrementos en número de vesículas listas para liberarse
3. 1 y 2 sugieren que xp es una variable importante para tratar de
elucidar mecanismos. I.e. sistema bidimensional para explicar
variabilidad en pcp

\textbf{Slide 3}
1. Dos variables de estado: x y p. Ambas tienden hacia un estado
estable, gobernadas cada por una constante de tiempo, y por una regla
de cambio (px para x, h(1-p) para p). 
2. Las variables en un principio pueden tener dinámica logística
(tasa de cambio sigmoidal)

\textbf{Slide 4}
