%& -shell-escape
%\documentclass[10pt,landscape]{article}
\documentclass[10pt]{article}
\usepackage[utf8]{inputenc}
\usepackage[spanish,es-nodecimaldot]{babel}
\usepackage[left=2.5cm,right=2.5cm,top=0cm,bottom=2cm,includehead]{geometry}


%%%%%%%%%%%%%%%%%%%%%%%%%%%
% Packages
%%%%%%%%%%%%%%%%%%%%%%%%%%%
\usepackage{hyperref}
\hypersetup{
  pdfauthor={Marco Arieli Herrera-Valdez},
  pdftitle={}
  pdftex,
  colorlinks=true,
  urlcolor=Bittersweet,
  linkcolor=blue,
  pdftoolbar=true,
  pdfmenubar=true,
  citecolor=Purple,
  filecolor=blue, 
}
%\usepackage{authblk}
\usepackage{python}
\usepackage{mathtools,amsfonts,amssymb,amsmath}
\usepackage[dvipsnames,svgnames,hyperref,table]{xcolor}
\usepackage{graphicx}
\usepackage{microtype}
\usepackage{sidecap}
%\hypersetup{pdfpagemode=FullScreen}
%\usepackage[left=2.5cm,right=2.5cm,top=0cm,bottom=2cm,includehead]{geometry}
\usepackage{fancyhdr}
%\pagestyle{fancyplain}
%\pagestyle{plain}
\usepackage{lscape}
\usepackage{setspace} 
%\setstretch{1.1}
%\doublespacing
%\usepackage[spanish,es-nodecimaldot]{babel}
\usepackage[utf8]{inputenc}
%\usepackage[latin1]{inputenc}
%\usepackage[applemac]{inputenc}

%% Only if the base font of the document is to be different, say sans serif
% Text layout
\usepackage[T1]{fontenc}
\usepackage[scaled=0.92]{helvet}
\renewcommand*\familydefault{\sfdefault}
%\topmargin 0.0cm
%\oddsidemargin 0.5cm
%\evensidemargin 0.5cm
%\textwidth 16cm 
%\textheight 21cm
% The color packages must appear before the pdfpages package
%\usepackage{chemarr}
\usepackage{listings}
%\usepackage[normalem]{ulem}
%\usepackage[usenames,svgnames,dvipsnames]{xcolor}
%\usepackage[dvipsnames,svgnames,usenames]{xcolor}
%\usepackage{booktabs} % Top and bottom rules for table
\usepackage[font=small,labelfont=bf]{caption} 
%\usepackage{wrapfig}
%\usepackage{subfigure}
%\usepackage{beamerthemesplit}
\usepackage{multirow}
\usepackage{multicol}
\usepackage{longtable}
\usepackage{times}
\usepackage{animate}
\usepackage{pdfpages}
\usepackage{url}
%\usepackage{multimedia}
%\usepackage{movie15} 
%\usepackage{media9} 
\usepackage{verbatim}
%\usepackage{pgflibraryarrows} 
%\usepackage{pgflibraryshapes}
\usepackage{tikz}
%\usetikzlibrary{arrows,shapes,matrix,chains,calc,positioning} 
%\usetikzlibrary{trees,mindmap} 
\usepackage{ifthen}
\usepackage{animate}

% To get the envelope in the author list
\usepackage[misc]{ifsym}
%\usepackage[misc,geometry]{ifsym}

%\Letter after the name of the corresponding author

% --------------------------------------
% Bibliography
% --------------------------------------
%\usepackage[sort&compress]{natbib} 
\usepackage[round,sort&compress]{natbib} 
%\usepackage[numbers,sort&compress]{natbib} 
%\bibliographystyle{plainnat}


% ------------------------------------------------
% Abbreviations and other commands
% ------------------------------------------------
\newcommand{\email}{\textrm{\Letter}}
\newcommand{\corrAuthor}{\textrm{\Letter}}
\newcommand{\insp}{{National Institute of Public Health}}
\newcommand{\cisei}{{Center for Research on Infectious Diseases}}
\newcommand{\imate}{{Mathematics Institute}}
\newcommand{\unam}{{National Autonomous University of Mexico}}
\newcommand{\ua}{{University of  Arizona}}
\newcommand{\uprc}{{University of Puerto Rico in Cayey}}
\newcommand{\mbi}{{Mathematical Biosciences Institute}}
\newcommand{\osu}{{Ohio State University}}
\newcommand{\asu}{{Arizona State University}}
\newcommand{\sols}{{School of Life Sciences}}
\newcommand{\smss}{{School of Mathematical Sciences and Statistics}}
\newcommand{\mcmsc}{{Mathematical, Computational, and Modelling Sciences Center}}
\newcommand{\shesc}{School of Human Evolution and Social Change}
%
 \newcommand{\INSP}{{Instituto Nacional de Salud Pública}}
\newcommand{\CISEI}{{Centro de Investigación Sobre Enfermedades Infecciosas}}
\newcommand{\UNAM}{{Universidad Nacional Aut\'onoma de M\'exico}}
\newcommand{\IM}{{Instituto de Matem\'aticas}}
\newcommand{\IF}{{Instituto de F\'isica}}
\newcommand{\FC}{{Facultad de Ciencias}}
\newcommand{\DM}{{Departamento de Matem\'aticas}}
% \newcommand{\ua}{{Universidad de  Arizona}}
% \newcommand{\asu}{{Universidad Estatal de Arizona}}
% \newcommand{\uprc}{{Universidad de Puerto Rico en Cayey}}
% \newcommand{\mbi}{{Instituto de Biociencias Matem\UTF{00C3}\UTF{00A1}ticas}}
% \newcommand{\osu}{{Universidad Estatal de Ohio}}
% ------------------------------------------------
\newcommand{\lrRound}[1]{\left(#1\right)}
\newcommand{\lrSquare}[1]{\left[#1\right]}
\newcommand{\lrSet}[1]{\left\{#1\right\}}
\newcommand{\lrAbs}[1]{\left|#1\right|}
\newcommand{\lrNorm}[1]{\left\|#1\right\|}
\newcommand{\prob}[1]{\mathbf{P}\left\{ #1\right\}}
\newcommand{\eg}{\textit{e.g.}}
\newcommand{\ie}{\textit{i.e.}}
\newcommand{\potassium}{{K$^+$}}
\newcommand{\kalium}{{K$^+$}}
\newcommand{\hydrogen}{{H$^+$}}
\newcommand{\sodium}{{Na$^+$}}
\newcommand{\natrium}{{Na$^+$}}
\newcommand{\calcium}{{Ca$^{2+}$}}
\newcommand{\chloride}{{Cl$^{-}$}}
\newcommand{\magnessium}{{Mg$^{2+}$}}
\newcommand{\concNa}{[Na]}
\newcommand{\concCa}{[Ca]}
\newcommand{\concCl}{[Cl]}
\newcommand{\concK}{[K]}
\newcommand{\felis}{{{$I_{cyc}$}}}
\newcommand{\icyc}{{{$I_{cyc}$}}}
\newcommand{\Avogadro}{N_{\mathrm{A}}}
\newcommand{\absTemp}{\mathrm{T}}
\newcommand{\GasConstant}{\mathrm{R}}
\newcommand{\Faraday}{\mathrm{F}}
\newcommand{\hPlanck}{\mathrm{h}}
\newcommand{\kBoltzmann}{\mathrm{k}}
\newcommand{\kT}{\mathrm{kT}}
\newcommand{\qElementary}{\mathrm{q}}
\newcommand{\hqovertwokT}[1]{\frac{\mathrm{e_0}#1}{2\mathrm{kT}}}
\newcommand{\hqoverkT}[1]{\frac{\mathrm{e_0}#1}{\mathrm{kT}}}
\def\qovertwokT{\mathrm{\frac{e_0}{2kT}}}
\def\qoverkT{\mathrm{\frac{e_0}{kT}}}
\newcommand{\trace}[1]{{\mathrm{Tr}(#1)}}
\newcommand{\tSm}{{\mathrm{Sm}}}
\newcommand{\tSy}{{\mathrm{Sy}}}
\newcommand{\tSt}{{\mathrm{St}}}
\newcommand{\tF}{{\mathrm{F}}}
\newcommand{\tLFP}{{\mathrm{LFP}}}
\newcommand{\tATP}{{\mathrm{ATP}}}
\newcommand{\tADP}{{\mathrm{ADP}}}
\newcommand{\tCaATP}{{\mathrm{CaATP}}}
\newcommand{\tP}{{\mathrm{P}}}
\newcommand{\tNa}{{\mathrm{Na}}}
\newcommand{\tNaK}{{\mathrm{NaK}}}
\newcommand{\tNaKa}{{\mathrm{NaKa}}}
\newcommand{\tNaP}{{\mathrm{NaP}}}
\newcommand{\tNaT}{{\mathrm{NaT}}}
\newcommand{\tNaH}{{\mathrm{NaH}}}
\newcommand{\tNaCa}{{\mathrm{NaCa}}}
\newcommand{\tNaKCl}{{\mathrm{NaKCl}}}
\newcommand{\tKCl}{{\mathrm{KCl}}}
\newcommand{\tCa}{{\mathrm{Ca}}}
\newcommand{\tCl}{{\mathrm{Cl}}}
\newcommand{\tCaL}{{\mathrm{CaL}}}
\newcommand{\tE}{{\mathrm{E}}}
\newcommand{\tI}{{\mathrm{I}}}
\newcommand{\tGlu}{{\mathrm{Glu}}}
\newcommand{\tGABA}{{\mathrm{GABA}}}
\newcommand{\tACh}{{\mathrm{ACh}}}
\newcommand{\tS}{{\mathrm{S}}}
\newcommand{\tSK}{{\mathrm{SK}}}
\newcommand{\tH}{{\mathrm{H}}}
\newcommand{\tK}{{\mathrm{K}}}
\newcommand{\tKaD}{{\mathrm{KaD}}}
\newcommand{\tKD}{{\mathrm{KD}}}
\newcommand{\tL}{{\mathrm{L}}}
\newcommand{\INa}{{I_\mathrm{Na}}}
\newcommand{\INaT}{{I_\mathrm{NaT}}}
\newcommand{\ICa}{{I_{\mathrm{Ca}}}}
\newcommand{\ICaL}{{I_{\mathrm{CaL}}}}
\newcommand{\IKD}{{I_{\mathrm{KD}}}}
\newcommand{\IKaD}{{I_{\mathrm{KaD}}}}
\newcommand{\IK}{{I_{\mathrm{K}}}}
\newcommand{\IAMPA}{{I_{\mathrm{AMPA}}}}
\newcommand{\INMDA}{{I_{\mathrm{NMDA}}}}
\newcommand{\IGABA}{{I_{\mathrm{GABA}}}}
\newcommand{\INaKATP}{{I_{\mathrm{NaK}}}}
\newcommand{\INaK}{{I_{\mathrm{NaK}}}}
\newcommand{\INaKa}{{I_{\mathrm{NaKa}}}}	
\newcommand{\INaCaX}{{I_{\mathrm{NaCa}}}}
\newcommand{\ILeak}{{I_{\mathrm{L}}}}
\newcommand{\stimtop}{{\textit{Top}}}
\newcommand{\stimup}{{\textit{Up}}}
\newcommand{\stimdown}{{\textit{Down}}}
\newcommand{\diffop}[2]{{\partial_{#1} #2}}
%
\newcommand{\splitPage}[4]{
\begin{minipage}{#1\textwidth}
{#2}
\end{minipage}%
\begin{minipage}{#3\textwidth}
{#4}
\end{minipage}
}

\newcommand{\rfig}[1]{{Fig.~\ref{#1}}}
\newcommand{\rtwofigs}[2]{{Figs.~\ref{#1}~and~\ref{#2}}}
\newcommand{\rfigs}[2]{{Figs.~\ref{#1}-\ref{#2}}}
\newcommand{\req}[1]{{Eq.~\eqref{#1}}}
\newcommand{\rtwoeqs}[2]{{Eqs.~\eqref{#1}~and~\eqref{#2}}}
\newcommand{\reqs}[2]{{Eqs.~\eqref{#1}-\eqref{#2}}}
\newcommand{\eqn}[1]{\begin{equation}#1\end{equation}}
\newcommand{\eqna}[1]{\begin{eqnarray}#1\end{eqnarray}}
\newcommand{\smalleq}[1]
{\begin{equation}{\small #1}\end{equation}}
\newcommand{\smalleqna}[1]{ 
  \begin{small} 
    \begin{eqnarray} #1 \end{eqnarray} 
  \end{small}
}
\newcommand{\tinyeqna}[1]{ 
  \begin{tiny} 
    \begin{eqnarray} #1 \end{eqnarray} 
  \end{tiny}
}
% LaTeX beamer
\newcommand{\slide}[1]{\begin{frame}#1\end{frame}}
\newcommand{\from}[1]{{\tiny From #1.}}
\newcommand{\fuente}[1]{{\tiny #1}}
\newcommand{\source}[1]{{\tiny #1}}
\newcommand{\sS}[3]{{{#1}_{#2}^{#3}}}


%%%%%%%%%%%%%%%%%%%%%%%%%%%
% Edits and reviews
%%%%%%%%%%%%%%%%%%%%%%%%%%%
\newcommand{\aquimequede}{
\vspace{1cm} \textcolor{red}
{\textbf{AQUI ME QUEDE}} \\
\vspace{1cm}}
\newcommand{\mahv}[1]{\textcolor{gray}{$^{MAHV:}$}\textcolor{orange}{#1} }




%%%%%%%%%%%%%%%%%%%%%%%%%%%
% Colored sections
%%%%%%%%%%%%%%%%%%%%%%%%%%%
\newcommand{\headcolor}[1]{\textcolor{Bittersweet}{#1}}
\newcommand{\colorsection}[1]{\section*{\headcolor{#1}}}
\newcommand{\colorsubsection}[1]{\subsection*{\headcolor{#1}}}
\newcommand{\colorparagraph}[1]{\paragraph{\headcolor{#1}}}
\definecolor{lightGray}{rgb}{0.92,0.92,0.92}


%%%%%%%%%%%%%%%%%%%%%%%%%%%
% Figure settings and inclusion/exclusion of figures from the compilation
%%%%%%%%%%%%%%%%%%%%%%%%%%%

\newcommand{\figu}[3]{
\begin{figure}{\begin{center}#3\end{center}}\caption{#1}\label{#2}\end{figure}}
\newcommand{\tablFigu}[3]{
\begin{table}\caption{#1}\label{#2}\begin{center}{#3}\end{center}\end{table}}
\newcommand{\smallcaption}[1]{\caption{{\small#1}}}
%\renewcommand{\caption}[1]{\caption{\begin{small}#1\end{small}}}
\usepackage{comment}
%\excludecomment{cond}
\includecomment{cond}
\newcommand{\showfigs}[1]{}
\begin{cond}
\renewcommand{\showfigs}[1]{\begin{center}#1\end{center}}
\end{cond}

% --------------------------------------
% mahvMacros
% --------------------------------------
% For supplements
% --------------------------------------
%\newcommand{\hbAppendixPrefix}{A}
%
%\renewcommand{\thefigure}{\hbAppendixPrefix\arabic{figure}}
%\setcounter{figure}{0}
%\renewcommand{\thetable}{\hbAppendixPrefix\arabic{table}} 
%\setcounter{table}{0}
%\renewcommand{\theequation}{\hbAppendixPrefix\arabic{equation}} 
%\setcounter{equation}{0}


\newenvironment{suppFig}[4]{
  \begin{figure}[h]
    \begin{center}
      {\includegraphics[#4]{#3}} \caption{\small #1} \label{#2}
    \end{center}
  \end{figure}}

\newenvironment{tab}[3]{
\begin{table}
\begin{center}
{#3} \caption{\small #1} \label{#2}
\end{center}
\end{table}}

% Operators
\DeclareMathOperator{\Tr}{Tr}
\newcommand{\re}[1]{ {\mathfrak{Re}} \left( {#1} \right)}
\newcommand{\im}[1]{ {\mathfrak{Im}} \left( {#1} \right)}
% Transpose upperscript
\newcommand{\transp}[1]{#1^{\mathsmaller T}}

% Column vectors. Usage:
% \colvector{a\\b\\c\\d\\e}
\newcommand{\colvector}[1]{\begin{pmatrix}#1\end{pmatrix}} 

% Column vectors. Usage:
% \colvec{5}{a}{b}{c}{d}{e}
\newcount\colveccount
\newcommand*\colvec[1]{
        \global\colveccount#1
        \begin{pmatrix}
        \colvecnext
}
\def\colvecnext#1{
        #1
        \global\advance\colveccount-1
        \ifnum\colveccount>0
                \\
                \expandafter\colvecnext
        \else
                \end{pmatrix}
        \fi
}

% ----------------------------------
% Code listing 
% ----------------------------------
\lstdefinestyle{customPython}{
  language=Python,
  belowcaptionskip=1\baselineskip,
  breaklines=true,
  %frame=L,
  xleftmargin=\parindent,
  showstringspaces=false,
  basicstyle=\footnotesize\ttfamily,
  keywordstyle=\bfseries\color{green!40!black},
  commentstyle=\itshape\color{purple!40!black},
  identifierstyle=\color{blue},
  stringstyle=\color{orange},
}


\hypersetup{
  pdftitle={Herrera-Valdez et al. Simple model for short-term
    plasticity}
}
%\usepackage{python}

% -----------------------------------------------
\title{Geometry and dynamics underlying the diversity of short term synaptic plasticities  GABAergic and glutamatergic terminals in striatal neurons}

\author{
Marco Arieli Herrera Valdez$^1$, Guillermo Olicón$^1$,\\  
Mario Alberto Arias-García$^2$, Janet Barroso$^2$,  José Bargas$^2$, Elvira Galarraga $^2$\\
\begin{small}
{$^1$Departamento de Matemáticas, Facultad de Ciencias, 
$^2$División de Neurociencias, Instituto de Fisiología Celular, U.N.A.M.}
\end{small}
}

\date{Agosto 21, 2016}

% -----------------------------------------------
% -----------------------------------------------
% -----------------------------------------------
% -----------------------------------------------
\begin{document}
\maketitle

\section{General rationale}

Define a synaptic contact between two neurons A and B as the set of all synapses where the presynaptic membrane is part of $A$ and the postsynaptic membrane is part of $B$. Each synapse is assumed to have a probability of release that depends on the amount of intracellular calcium $c$ as 
\begin{equation}
%p (c;g,c_h)= \frac{1}{1+ \exp \lrSquare{ g_c \lrRound{c_{h}-c} }}
p (c;a,c_h)= \frac{c^{a}}{c^{a}+c_h^{a}}
\label{pSynRelease}
\end{equation}
where $c_{h}$ is the half-maximal calcium concentration and $g_c$ is a gain parameter.  

The calcium dynamics for a single terminal can be described as 
\begin{equation}
\partial_t c = \frac{c_{\infty}-c}{\tau_c} - k_c I_c(t)
\label{eq:partialtc}
\end{equation}
where $c_{\infty}$ is the steady state concentration of intracellular calcium, $I_c$ is the calcium current at the terminal, and $k_c$ is a conversion factor. 
The range for $c_{\infty}$ can be assumed to be between 0.1 and 0.5 $\mu$M. 

It is possible to derive an equation that describes the dynamics of the probability of release can then be derived taking into consideration equations~\eqref{pSynRelease} and~\eqref{eq:partialtc}. By the chain rule, 
\begin{eqnarray*}
\partial_t p &=& \partial_c p \cdot \partial_t c 
\\
&=& \frac{a}{c} p(c) \lrRound{1-p(c)}  \lrRound{\frac{c_{\infty}-c}{\tau_c} - k_c I_c(t)}
\end{eqnarray*}
However, if the dynamics for $p$ are fast enough, then it should be possible to assume $p = p_{\infty}$.

% ------------------------------------------
\section{Peaks of the pre- and post-synaptic responses to trains of presynaptic action potentials}

Assume that presynaptic action potentials occur at times $0<t_1<t_2<\ldots<t_n$. Let 
$x(t)$ and $p(t)$ represent the readily releasable neurotransmitter and the proportion of released neurotransmitter at time $t$, respectively.
Recall that the dynamics for $p$ and $x$ can be written as
\begin{eqnarray*}
\label{modelo}
\partial_t p &=& \frac{p_{\infty}-p}{\tau_p} + a(1-p) \phi(t)
\\
\partial_t x &=& \frac{x_{\infty}-x}{\tau_x} - px \phi(t)
\\
\phi(t) &=& \sum_{k=1}^{n} \delta\lrRound{t-t_k}
\end{eqnarray*}
In the absence of a pulse the dynamics of $p$ and $x$ are given by
\begin{eqnarray*}
\label{pxNoPulseDynamics}
p(t)&=&p_{\infty}+(p_0-p_{\infty})\exp{\left(\frac{t_0-t}{\tau_p}\right)}
\\
x(t)&=&x_{\infty}+(x_0-x_{\infty})\exp{\left(\frac{t_0-t}{\tau_x}\right)}
\end{eqnarray*}
where $p(t_0)=p_0$ and $x(t_0)=x_0$ are initial conditions for $p$ and $x$. 
At each pulse, $p$ and $x$ jump to new initial conditions.  Specifically, 
at time $t=t_i$, $p$ changes to $a (1-p)$. As a consequence of the change in $p$, $x(t_i)$ changes to $x_i = x(t_i) (1- p_i)$. The amount of neurotransmitter released at time $t=t_i$ is then $x(t_i)- x_i = x(t_i) p_i$.

Assuming that the initial condition $p_0=p_{\infty}$, we have $p(t_1)=p_{\infty}$. We can now consider a new initial condition due to the effect of the term $a(1-p)\phi (t)$ given by
\[p_1=p(t_1)+a(1-p(t_1)).\]
In general, the solution for $p$ in each interval $\left[t_n,t_{n+1}\right]$ can be obtained by taking $p^*=p(t_n)$ and $t^*=t_n$ in (\ref{sol1}).

The relation between $p(t_k)$ and the new initial condition at times $t=t_k$ is given by the non-autonomous difference equation
\begin{equation}
    \label{DE}
        \begin{array}{rcl}
             p(t_{k+1})&=&p_{\infty} + (p_k-p_{\infty})\exp{\left(\frac{t_k-t_{k+1}}{\tau_p}\right)} \\
             p_{k+1}& = & p(t_{k+1}) + a (1-p(t_{k+1})).
        \end{array}
\end{equation}

It can be proved by induction that for $n\geq2$
\begin{eqnarray*}
p(t_n)&=&p_{\infty} + a(1-p_{\infty})\sum_{m=0}^{n-2}(1-a)^m \exp{\left(\frac{t_{n-m-1}-t_{n}}{\tau_p}\right)}
\\ 
p_{n} &=& p_{\infty} + a \lrRound{1-p_{\infty}} \sum_{m=0}^{n} (1-a)^m \exp \lrRound{\frac{t_{n-m}- t_{n}}{\tau_p}} 
\end{eqnarray*}

In a similar way, assuming that the initial condition for $x(t)$ lies in the stationary state $x_{\infty}$ one has that $x(t_1)=x_{\infty}$. The jump due to $\phi(t)$ can be measured as $x_1=x(t_1)(1-p_1)$, and in general the quantities $x(t_n)$ and $x_n$ satisfy the non-autonomous difference equation
\begin{eqnarray*}
    x(t_{n+1}) & = & x_{\infty} - (x_k-x_{\infty})\exp{\left(\frac{t_k-t_{k+1}}{\tau_x}\right)}
    \\
    x_{k+1} & = & x(t_{n+1}) - p_{n+1}x(t_{n+1})
\end{eqnarray*}

After some cumbersome algebra, an inductive argument yields for $n\geq2$
\begin{eqnarray*}
x_{n} 
&=& x_{\infty} \lrRound{1-p_n}
\lrSquare{
1 - p_{n-1}\exp{\frac{t_{n-1}-t_{n}}{\tau_x}} 
- \sum_{k=2}^{n-1} p_{n-k} \prod_{n-k+1}^{n-1} \lrRound{1-p_j} \exp\lrRound{\frac{t_{n-k}-t_{n}}{\tau_x}}
}
\\
&=& x_{\infty} 
\lrSquare{
\lrRound{1-p_n} - \sum_{k=1}^{n-1} p_{n-k} 
\prod_{n-k+1}^{n} \lrRound{1-p_j} \exp\lrRound{\frac{t_{n-k}-t_{n}}{\tau_x}}
}
\end{eqnarray*}

The dynamics of the proportion of activated postsynaptic channels, as proposed by \cite{destexhe1998kinetic}
\begin{eqnarray*}
\partial_t y &=& y\lrSquare{\alpha \eta(t) \lrRound{1 -y} - \beta y} 
\end{eqnarray*}
where $\eta(t)= \delta(t-t_i) x(t_i) p_i$.



In the absence of a complete model that generates action potentials, the time course of the calcium current at the terminal can be assumed to be a sum of pulses, each of which behaving like $\alpha$-functions  \citep{}:
\begin{equation}
\alpha(t;\tau) = 1_{\lrSet{t\geq0}} \frac{t}{\tau} e^{1-\frac{t}{\tau}}
\end{equation}

\subsection{Pulses of intracellular calcium concentration}

\begin{eqnarray}
\partial_t c &=& \frac{c_{\infty} -c}{\tau_c}+ \phi(t)
\\
\phi(t) &=& a \sum_{i=1}^{n} \delta\lrRound{t-t_i}
\end{eqnarray}
The dynamics in the absence of pulses are given by
\begin{eqnarray}
c(t) &=& c_{\infty} + \lrRound{c_0-c_{\infty}} \exp \lrRound{-\frac{t}{\tau_c}}  
\end{eqnarray}
If a pulse occurs at time $t$, then $c(t)$ jumps to $c(t) + a$. 
Let  $c_0 = c_{\infty}$. If a pulse occurs at time $t=t_1$, then let
\begin{eqnarray}
c_1 &=& c(t_1) + a = c_{\infty} +a
\end{eqnarray}
For $t_1\leq t $:
\begin{eqnarray}
c(t) &=& c_{\infty} + \lrRound{c_1-c_{\infty}} \exp \lrRound{\frac{t_1-t}{\tau_c}} 
\\
 &=& c_{\infty} + a  \exp \lrRound{\frac{t_1-t}{\tau_c}} 
\end{eqnarray}
\begin{eqnarray*}
c_2 &=& 
c(t_2) + a 
\\
&=&
c_{\infty} + a \lrRound{1 + \exp \lrRound{\frac{t_1 -t_2}{\tau_c}} }
\end{eqnarray*}
For $t_2\leq t $:
\begin{eqnarray*}
c(t) &=& 
c_{\infty} + \lrRound{c_2-c_{\infty}} \exp \lrRound{\frac{t_2-t}{\tau_c}} 
\\
 &=& c_{\infty} + a \lrRound{1 + \exp \lrRound{\frac{t_1 -t_2}{\tau_c}} }  \exp \lrRound{\frac{t_2-t}{\tau_c}} 
 \\
 &=& c_{\infty} + a \lrSquare{\exp \lrRound{\frac{t_2-t}{\tau_c}}  + \exp \lrRound{\frac{t_1 -t}{\tau_c}} }  
\end{eqnarray*}
At $t=t_3$,
\begin{eqnarray*}
c_3 &=& 
c(t_3) + a 
\\
&=&
c_{\infty} + a \lrRound{1 + \exp \lrRound{\frac{t_2-t_3}{\tau_c}}  + \exp \lrRound{\frac{t_1 -t_3}{\tau_c}}}
\end{eqnarray*}
In general, for $t\geq t_n$,
\begin{eqnarray*}
c(t) &=& 
c_{\infty} + \lrRound{c_n -c_{\infty}} \exp \lrRound{\frac{t_2-t}{\tau_c}} 
\\
 &=& c_{\infty} + a \sum_{k=1}^{n}  \exp \lrRound{\frac{t_k -t}{\tau_c}}
 \\
 &=& c_{\infty} + a \exp \lrRound{\frac{ -t}{\tau_c}} \sum_{k=1}^{n}  \exp \lrRound{\frac{t_k }{\tau_c}}
\end{eqnarray*}



Assume that a neuron receives $M$ synaptic contacts of the same type (e.g. AMPA, GABA-A).

\section{Linear dynamics}

Let us assume that a neuron receives $N$ synapses, with $n_j$ readily releasable vesicles and $p_j$ the probability of release at the $j$th synapse.  
Assume that all the synapses are all GABAergic or all glutamatergic, let $p = N^{-1}\sum_{i=1}^N p_i$ and $x = N^{-1}\sum_{i=1}^N x_i$ represent  the average of the 

\begin{eqnarray}
\partial_t p &=&  \frac{p_{\infty} - p}{\tau_p} + h (1-p) \sum \delta\lrRound{t-t_i}
\\
\partial_t x &=&  \frac{x_{\infty} - x}{\tau_x} -  px \sum \delta\lrRound{t-t_i}
\end{eqnarray}

\subsection{Analysis for $p$}. 
\begin{eqnarray}
p (t) 
&=& \frac{p_{\infty} p_0}{p_0 + \lrRound{p_{\infty} - p_0} \exp
  \lrRound{\frac{ -t p_{\infty} }{\tau_p}} } 
\\
&=& {p_{\infty} }\lrSquare{1 + \lrRound{\frac{p_{\infty} }{p_0} -1} \exp
  \lrRound{\frac{ -t p_{\infty} }{\tau_p}} } ^{-1}
\\
&=& {p_{\infty} }\lrSquare{1 - \lrRound{1- \frac{p_{\infty} }{p_0} } \exp
  \lrRound{\frac{ -t p_{\infty} }{\tau_p}} } ^{-1}
\end{eqnarray}

If $p(0)=p_{\infty}$ and there is a pulse at time $t=t_0$, then
$p(t_0)=p_{\infty} + h \lrRound{1-p_{\infty}} 
= p_{\infty}\lrRound{1-h} + h$. Before the next pulse at time $t_0<t<t_1$, 
\begin{eqnarray}
p (t) &=& {p_{\infty} }\lrSquare{1 + \lrRound{\frac{p_{\infty} }{p_{\infty} -1} \lrRound{1-h} + h} \exp
  \lrRound{\frac{ -t p_{\infty} }{\tau_p}} } ^{-1}
\end{eqnarray}





\section{Logistic dynamics}
\begin{eqnarray}
\partial_t p &=& p \frac{p_{\infty} - p}{\tau_p} + h (1-p) \sum \delta\lrRound{t-t_i}
\\
\partial_t x &=& x \frac{x_{\infty} - x}{\tau_x} -  px \sum \delta\lrRound{t-t_i}
\end{eqnarray}

Analysis for $p$. 
\begin{eqnarray}
p (t) 
&=& \frac{p_{\infty} p_0}{p_0 + \lrRound{p_{\infty} - p_0} \exp
  \lrRound{\frac{ -t p_{\infty} }{\tau_p}} } 
\\
&=& {p_{\infty} }\lrSquare{1 + \lrRound{\frac{p_{\infty} }{p_0} -1} \exp
  \lrRound{\frac{ -t p_{\infty} }{\tau_p}} } ^{-1}
\\
&=& {p_{\infty} }\lrSquare{1 - \lrRound{1- \frac{p_{\infty} }{p_0} } \exp
  \lrRound{\frac{ -t p_{\infty} }{\tau_p}} } ^{-1}
\end{eqnarray}

If $p(0)=p_{\infty}$ and there is a pulse at time $t=t_0$, then
$p(t_0)=p_{\infty} + h \lrRound{1-p_{\infty}} 
= p_{\infty}\lrRound{1-h} + h$. Before the next pulse at time $t_0<t<t_1$, 
\begin{eqnarray}
p (t) &=& {p_{\infty} }\lrSquare{1 + \lrRound{\frac{p_{\infty} }{p_{\infty} -1} \lrRound{1-h} + h} \exp
  \lrRound{\frac{ -t p_{\infty} }{\tau_p}} } ^{-1}
\end{eqnarray}



\section{Log relationship between {\calcium} and probability of release}

Usually concentrations of $\concCa_i$ are expressed in $\log_{10}$ units.  The relationship between ${\concCa_i}$ and the probability of release in a presynaptic terminal is usually described by
\begin{eqnarray*}
p(c; c0, b) &=& \frac{\exp\lrSquare{b \lrRound{l-l_0}}}{1+\exp\lrSquare{b \lrRound{l-l_0}}}
\end{eqnarray*}
\begin{eqnarray*}
l= \log_{10} c &=& \lrRound{\log_{10} e }\lrRound{\ln c }
\end{eqnarray*}
Then 
\begin{eqnarray*}
e^l &=& \lrRound{e^{\ln c  }}^ {\log_{10} e  }
= c^{ \log_{10} e } 
\end{eqnarray*}
so
\begin{eqnarray*}
p(c; c_h, b) 
&=& \frac{c^{ b \log_{10} e } }{c^{ b \log_{10} e }+c_h^{ b \log_{10} e }}
\\
&=& \frac{c^{a} }{c^{a}+c_h^{a}}
\end{eqnarray*}
with $a=b \log_{10} e$.

\bibliographystyle{plainnat}
\bibliography{membraneBiophysics}
\end{document}
