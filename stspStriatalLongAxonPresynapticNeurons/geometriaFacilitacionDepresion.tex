%& -shell-escape
\documentclass[8pt]{beamer}
\usepackage[utf8]{inputenc}
\input{mahvPreambleBeamer}
\hypersetup{
  pdftitle={Herrera-Valdez et al. Simple model for short-term
    plasticity}
}
\usepackage{python}

\title[Geometría y dinámica de la plasticidad sináptica]{Geometría y
  dinámica detrás de la diversidad en plasticidad sináptica de corto
  plazo en terminales GABAérgicas del estriado}

\author[Marco Arieli Herrera Valdez]{Marco Arieli Herrera Valdez \\ 
\textit{Instituto de Matemáticas, U.N.A.M.}\\
\vspace{10pt}
\small{Colaboradores: Janet
    Barroso y José Bargas,\\
\textit{Instituto de Fisiología Celular, U.N.A.M.}}
}
\date{Octubre 22, 2014}

\begin{document}
% -----------------------------------------------
\begin{frame}
\maketitle
\end{frame}

\section{Simple model of short-term plasticity}

% -----------------------------------------------
\begin{frame}
\frametitle{Simple model of short-term plasticity}
Plasticity here means either depression or enhancement \citep{}.

Assume that a presynaptic cell fires action potentials at times $t_k$,
$k =1,2,...$. Let $x$ be the occupancy of the release pool and $p$ the
probability of release. The release dynamics in the presynaptic side
can then be described by 
\begin{eqnarray}
\partial_t p &=&\frac{p_{\infty}-p}{\tau_{f}} + \lrRound{1-p} \sum \textcolor{red}{q(h_k)}
\phi \lrRound{t,t_k,\mu,\sigma},
\label{eq:prelease}
\\
%\partial_t x &=& \frac{x\lrRound{x_{\infty}-x}}{\tau_{r}} - \sum \phi
%\lrRound{t,t_k,\mu,\sigma}  p x, 
\partial_t x &=& \frac{x\lrRound{1-x}}{\tau_{r}} - \sum \phi
\lrRound{t,t_k,\mu,\sigma}  p x, 
\label{eq:occupancy}
\\
%\delta(y) &=& 1 \textrm{ if } y=0, \quad 0 \textrm{ otherwise} 
\phi(t,t_k,\mu,\sigma) &=& \frac{1}{\sigma \sqrt{2 \pi}} \exp
\lrSquare{- \frac{\lrRound{t-t_k -3 \sigma}^2}{2  \sigma^2}},
\label{eq:phi}
\end{eqnarray}
where $h_k=t_{k} -t_{k-1}$ for $k=2,3,...$. The parameters $p_{\infty}$,
$x_{\infty}$, $\tau_{r}$, and $\tau_{f}$ represent the 
steady state probability of release, the steady state occupancy, and the recovery and facilitation and
recovery time constants, respectively. The increase in $p$, denoted by
\textcolor{red}{$q(h_k) \in (0,1)$} can be assumed to depend on the presynaptic
interspike intervals. For instance, 
\begin{equation}
q(h)= \bar{q} h   e^{1- h/\tau_q}
\end{equation}
where $\tau_q$ can be regarded as a facilitation time constant, and
$\bar{q}$ is the maximum increase in the probability of release. 
\end{frame}



% -----------------------------------------------
\begin{frame}
\frametitle{Analysis of pulsed activity for constant $q$ and periodic
  spiking with instantaneous jumps}

\begin{multicols}{2}
\begin{tiny}

  Let $(p(0),x(0))=(p_0,x_0)$.  Assume that $q \in [0,1]$ is constant
  and that action potentials occur at times $0<t_1, t_2,...,t_n$,
  which induce instantáneous jumps. That is, $\phi(t,t_k) =
  \delta(t-t_k)$, $k=1,...,n$.  Also, assume the interspike intervals
  $t_k - t_{k-1}=h$ are constant for $k=2,...,n$.
\begin{eqnarray}
\underbar{p}_k&=& \lim_{\epsilon\rightarrow 0} p(t_k-\epsilon), 
\\
\underbar{x}_k&=& \lim_{\epsilon\rightarrow 0} x(t_k-\epsilon),
\end{eqnarray}
for $k=1,...,n$. 
At the time of arrival for the $k$th pulse, the system jumps as follows:
\begin{eqnarray}
\underbar{p}_k& \mapsto & p_k= \underbar{p}_k + q (1-\underbar{p}_k) = \underbar{p}_k \lrRound{1-q} + q,
\\
\underbar{x}_k& \mapsto &
x_k
= \underbar{x}_k - \underbar{x}_k \underbar{p}_k 
= \underbar{x}_k \lrRound{1-\underbar{p}_k} ,
\end{eqnarray}
That is, $(p_k,x_k)$ is the state of the system after the $k$th
action potential occurs. 

The dynamics of $(p(t),x(t))$ at a time $t$ between pulses
linear and logistic dynamics, respectively, which yield functions of the form
\begin{eqnarray}
p(t) &=& p_{\infty} - \lrRound{p_{\infty}- p_k}
   \exp\lrRound{-\frac{(t-t_k)}{\tau_p}},
 \\
x(t) &=& x_{\infty} {x_k} \lrSquare{{x_k} + \lrRound{x_{\infty}- x_k}
\exp\lrRound{-\frac{(t-t_k)}{\tau_x}x_{\infty} }}^{-1},
\end{eqnarray}
for $k=1,...,n$ indexing the time of the last pulse.


Recall $(p(0),x(0))=(p_0,x_0)$. 
Written explicitly, the state of the system at
the arrival of the pulse at time $t_k$ is
\begin{eqnarray}
p_k &=& \lrSquare{p_{\infty} - \lrRound{p_{\infty}- p_{k-1}}
  e^{-\frac{(t_k-t_{k-1})}{\tau_p}}}\lrRound{1-q} + q,
\\
x_{k} &=& \frac
{{x_{k-1}} x_{\infty} \lrSquare{1-p_{\infty} + \lrRound{p_{\infty}- p_{k-1}}
  e^{-\frac{(t_k-t_{k-1})}{\tau_p}}}}
{ x_{k-1} + \lrRound{x_{\infty}- x_{k-1}} e^{-\frac{(t_k-t_{k-1})}{\tau_x}x_{\infty} }}
\end{eqnarray}
for $k=1,...,n$.  The states $(x_k,p_k)$, $k=0,...,n$ can thus be
regarded as an initial condition to  calculate the state of the system at any time $t$ in which there is
not an action potential. 

\end{tiny}
\end{multicols}
\end{frame}


\begin{frame}
\begin{tiny}
Explicitly, at the arrival of the first pulse, 
\begin{eqnarray}
p_1 &=& \lrSquare{p_{\infty} - \lrRound{p_{\infty}- p_0}
  e^{-\frac{t_1}{\tau_p}}}\lrRound{1-q} + q,
\\
x_{1} &=& \frac{{x_{0}} x_{\infty} \lrSquare{1-p_{\infty} + \lrRound{p_{\infty}- p_{0}}
  e^{-\frac{t_1}{\tau_p}}}}{ {x_{0}} + \lrRound{x_{\infty}- x_{0}}
  e^{-\frac{t_1}{\tau_x}x_{\infty} }}. 
\end{eqnarray}
After the second pulse,
\begin{eqnarray}
p_2 
&=& \lrSquare{p_{\infty} - \lrRound{p_{\infty}- \textcolor{red}{p_1}}
  e^{-\frac{t_2-t_1}{\tau_p}}}\lrRound{1-q} + q,
\nonumber \\
&=& \lrSquare{p_{\infty} - \lrRound{p_{\infty}  
\textcolor{red}{
-\lrRound{p_{\infty} - \lrRound{p_{\infty}- p_0}e^{-\frac{t_1}{\tau_p}}}\lrRound{1-q} - q}}
  e^{-\frac{t_2-t_1}{\tau_p}}}\lrRound{1-q} + q,
% \nonumber \\
% &=& \lrSet{p_{\infty}-
% \lrSquare{
% \lrRound{p_{\infty}- p_0} \exp\lrRound{-\frac{t_1}{\tau_p}}
% \lrRound{1-q}
% - q \lrRound{1-p_{\infty} }}
% \exp\lrRound{-\frac{t_2}{\tau_p}}} \lrRound{1-q} + q,
\\
x_{2} 
&=& \frac{{\textcolor{blue}{x_{1}}} x_{\infty} \lrSquare{1-p_{\infty} + \lrRound{p_{\infty}- \textcolor{red}{p_{1}}}
e^{-\frac{t_2-t_1}{\tau_p}}}}
{ {\textcolor{blue}{x_{1}}} + \lrRound{x_{\infty}- \textcolor{blue}{x_{1}}}
e^{-\frac{t_2-t_1}{\tau_x}x_{\infty} }}, 
\nonumber \\
&=& \frac{ \textcolor{blue}{\lrRound{\frac{{x_{0}}x_{\infty} \lrSquare{1-p_{\infty} + \lrRound{p_{\infty}- p_{0}}
  e^{-\frac{t_1}{\tau_p}}}}{ {x_{0}} + \lrRound{x_{\infty}- x_{0}}
  e^{-\frac{t_1}{\tau_x}x_{\infty} }}}} x_{\infty} \lrSquare{1-p_{\infty} + \lrRound{p_{\infty}- 
 \textcolor{red}{\lrSet{ \lrSquare{p_{\infty} - \lrRound{p_{\infty}- p_0}
 e^{-\frac{t_1}{\tau_p}}}\lrRound{1-q} + q}
}}  e^{-\frac{t_2-t_1}{\tau_p}}}}
{{\textcolor{blue}{\lrRound{\frac{{x_{0}} x_{\infty} \lrSquare{1-p_{\infty} + \lrRound{p_{\infty}- p_{0}}
  e^{-\frac{t_1}{\tau_p}}}}{ {x_{0}} + \lrRound{x_{\infty}- x_{0}}
  e^{-\frac{t_1}{\tau_x}x_{\infty} }}}}  + \lrRound{x_{\infty}-
\textcolor{blue}{\lrRound{\frac{{x_{0}} x_{\infty} \lrSquare{1-p_{\infty} + \lrRound{p_{\infty}- p_{0}}
  e^{-\frac{t_1}{\tau_p}}}}{ {x_{0}} + \lrRound{x_{\infty}- x_{0}}
  e^{-\frac{t_1}{\tau_x}x_{\infty} }}}}
}}
  e^{-\frac{t_2-t_1}{\tau_x}x_{\infty} }}
\end{eqnarray}

\end{tiny}
\end{frame}

% -----------------------------------------------
\begin{frame}
\begin{tiny}
After the third pulse:
\begin{eqnarray}
p_3 
&=& \lrSquare{p_{\infty} - \lrRound{p_{\infty}- \textcolor{red}{p_2}}
  \exp\lrRound{-\frac{t_3}{\tau_p}}}\lrRound{1-q} + q,
%\nonumber \\
\\
x_{3} 
&=& \frac{{\textcolor{blue}{x_{2}}} x_{\infty} \lrSquare{1-p_{\infty} + \lrRound{p_{\infty}- \textcolor{red}{p_{2}}}
e^{-\frac{t_3}{\tau_p}}}}
{ {\textcolor{blue}{x_{2}}} + \lrRound{x_{\infty}- \textcolor{blue}{x_{2}}}
e^{-\frac{t_3}{\tau_x}x_{\infty} }}, 
%\nonumber \\
\end{eqnarray}

\end{tiny}
\end{frame}


% -----------------------------------------------
% \begin{frame}
% \begin{center}
% \begin{tabular}{cc}
% Facilitation & Depression \\
% \includegraphics[width=0.45\textwidth]{figures/q0dot20omegaStim20Hz_tauRec15msec_tauFac75msec.png}
% & 
% \includegraphics[width=0.45\textwidth]{figures/q0dot20omegaStim20Hz_tauRec120msec_tauFac75msec.png}
% \\
% \includegraphics[width=0.45\textwidth]{figures/q0dot20omegaStim20Hz_tauRec60msec_tauFac75msec.png}
% &
% \includegraphics[width=0.45\textwidth]{figures/q0dot20omegaStim20Hz_tauRec150msec_tauFac75msec.png}
% \\
% \includegraphics[width=0.45\textwidth]{figures/q0dot20omegaStim20Hz_tauRec90msec_tauFac75msec.png}
% &
% \includegraphics[width=0.45\textwidth]{figures/q0dot20omegaStim20Hz_tauRec210msec_tauFac75msec.png}
% \end{tabular}
% \end{center}
% {\tiny Parameters: , $x_0$=1, $p_0$=0.2, $q$=0.2, $x_{\infty}$=1, $p_{\infty}$=0.2,
%   $\tau_{f}$ = 75 ms, $\omega_{stim}$=20 Hz.}
%\caption{Simulations of the system
%\eqref{eq:occupancy}-\eqref{eq:phi} with $p$-increments of size
%$q$=0.2. Other parameters specified in the figure labels.}
%\end{frame}



\end{document}
