%& -shell-escape
\documentclass[8pt]{beamer}
\usepackage[utf8]{inputenc}
\usepackage[spanish,es-nodecimaldot]{babel}
%\setbeamertemplate{note page}[plain]
%\setbeameroption{show notes on second screen}

\input{mahvPreambleBeamer}
\hypersetup{
  pdftitle={Herrera-Valdez et al. Simple model for short-term
    plasticity}
}
\usepackage{python}

% -----------------------------------------------
\title[Geometría y dinámica de la plasticidad sináptica]{Geometría y
  dinámica detrás de la diversidad en plasticidad sináptica de corto
  plazo en terminales GABAérgicas del estriado}

\author[Marco Arieli Herrera Valdez, marcoh@ciencias]{Marco Arieli Herrera Valdez \\ 
\textit{Matemáticas, Facultad de Ciencias, U.N.A.M.}\\
\vspace{10pt}
\small{Colaboradores: \\
Janet Barroso, Elvira Galarraga, José Bargas, 
\textit{Instituto de Fisiología Celular, U.N.A.M., \\
Guillermo Olicón, Imperial College, UK
}}
}
\date{Febrero 23, 2017}

% -----------------------------------------------
% -----------------------------------------------
% -----------------------------------------------
% -----------------------------------------------
\begin{document}



% -----------------------------------------------
\begin{frame}
\maketitle
\end{frame}

\section{Resúmen}
\begin{frame}

\textbf{Variedad en plasticidad sináptica como función del balance entre la ocupación y la probabilidad de liberación de vesículas.}

\vspace{13pt} La comunicación sináptica en terminales GABAérgicas
depende de factores que incluyen el número de vesículas listas para
liberarse y de su probabilidad de liberación.  Registros de corriente
postsináptica en sinapses de largo alcance dentro del estriado
muestran patrones de depresión, facilitación, o combinación de las
anteriores, que pueden, \textit{a priori}, asociarse a distintos tipos
de terminales en distintos tipos de interneuronas (Barroso \textit{et
  al.}). Para entender mejor los mechanismos detrás de los patrones
descritos anteriormente, planteamos un sistema dinámico que nos
permite clasificar, de manera funcional, las terminales descritas
anteriormente con base en combinaciones de parámetros asociados al
incremento en la probabilidad de liberación y la capacidad de
re-ocupación del depósito de vesículas listas para liberarse.
\end{frame}

% __________________________________________________
%
\section{Modelo simple de plasticidad a corto plazo}
% __________________________________________________


% oooooooooooooooooooooooooooooo
\titledSlide{Plasticidad sináptica de corto plazo}{
  \begin{minipage}{0.5\textwidth}
    \begin{center}
      \only<1>{\includegraphics[width=0.9\textwidth]{./figuresSynapse/hippocampalSTP+Fig2+Roy+etal+2013}
        \\    
        \fuente{Roy, \textit{et al.}, 2013}
      }
      \only<2->{\includegraphics[width=0.9\textwidth]{./figuresSynapse/synapseHenning2014Fig1}
        \\    
        \fuente{Henning, \textit{et al.}, 2014}
      }
    \end{center}
  \end{minipage}%
  \begin{minipage}{0.5\textwidth}
    \uncover<1-3>{\textcolor{blue}{Plasticidad a corto plazo (PCP)}:
      cambio (def. funcional) en la amplitud de la corriente post-synaptica como
      consecuencia de actividad reciente en las neuronas que forman la
      sinápsis.  }

    \vspace{10pt}
    \uncover<2-3>{\textcolor{blue}{PCP} depende de la estructura de la
      sinápsis, la arquitectura de la zona activa, los perfiles de
      concentración del {\calcium} en la terminal, la probabilidad de
      liberación de vesículas, la difusión de neurotransmisor en el
      hueco sináptico, y la activación de receptores post-sinápticos.}
    
    \vspace{10pt}
    \uncover<3->{Es  \textbf{difícil}, o
      \textit{imposible}  estimar parámetros para modelos
      biofísicos-bioquímicos detallados por limitaciones técnicas, y
      los modelos existentes son complejos y difíciles de analizar.}

  \end{minipage}
}

% oooooooooooooooooooooooooooooo
\section{Plasticidad sináptica}


\begin{frame}
\frametitle{Interconectividad de largo alcance dentro del estriado}
\begin{center}
\begin{tabular}{cc}
\includegraphics[height=0.5\textheight]{./figuresSynapse/Fig-2-Local-interactions-of-striatal-interneurons-Scheme}
&
\includegraphics[height=0.5\textheight]{./figuresSynapse/Fig-5-Synergistic-effect-of-the-cell-speci-fi-city-of-corticostriatal-STDPs-on-the}
\\
&
\source{Fino \& Venance, 2011}
\end{tabular}
\end{center}
\end{frame}


% oooooooooooooooooooooooooooooo
\begin{frame}
\frametitle{Interconectividad de largo alcance dentro del estriado}
\begin{center}
\begin{tabular}{cc}
\includegraphics[height=0.5\textheight]{./figuresSynapse/striatalInterneuronLongRangeConnection2}
&
\includegraphics[height=0.5\textheight]{./figuresSynapse/striatalInterneuronLongRangeConnection1}
\\
\source{Rafalovich, et al. 2015}
&
\source{Tepper, Tecuapetla, Koós, Ibañez-Sandoval, 2010}
\end{tabular}
\end{center}
\end{frame}


% oooooooooooooooooooooooooooooo\
\begin{frame}
\frametitle{Perfiles de plasticidad sináptica en neuronas espinosas del estriado}
\begin{center}
\includegraphics[height=0.9\textheight]{./figuresSynapse/Barroso-Flores+etal+Fig1}
\\
\source{Barroso-Flores, Herrera-Valdez, López-Huerta, Galarraga, Bargas, 2015}
\end{center}
\end{frame}

% oooooooooooooooooooooooooooooo
\begin{frame}
\frametitle{Respuestas postsinápticas como función de la intensidad del estímulo y reclutamiento}
\begin{center}
\includegraphics[width=0.9\textwidth]{./figuresSynapse/Barroso-Flores+etal+Fig2}
\end{center}
\end{frame}

% oooooooooooooooooooooooooooooo
\begin{frame}
\begin{center}
\includegraphics[width=0.7\textwidth]{./figuresSynapse/dopamineDepletionConnectivityChanges1}
\\
\vspace{0.1\textheight}
\uncover<2>{
¿Qué pasa con la plasticidad sináptica en enfermos de Parkinson?
}
\end{center}
\end{frame}

% oooooooooooooooooooooooooooooo
\begin{frame}
\frametitle{Lesiones para inducir síntomas de Parkinson}
\begin{center}
\begin{tabular}{p{0.4\textwidth} p{0.4\textwidth}}
Lesión unilateral en SN/VTA 
& Prueba de comportamiento 2 semanas después
\end{tabular}
\includegraphics[width=\textwidth]{./figuresSynapse/Barroso-Flores+etal+Fig3}
\end{center}
\end{frame}

% oooooooooooooooooooooooooooooo
\begin{frame}
\frametitle{Cambios en plasticidad en respuesta a la falta de dopamina, p1}
\begin{center}
\includegraphics[height=0.9\textheight]{./figuresSynapse/Barroso-Flores+etal+Fig4}
\end{center}
\end{frame}

% oooooooooooooooooooooooooooooo
\begin{frame}
\frametitle{Cambios en plasticidad en respuesta a la falta de dopamina, p2}
\begin{center}
\includegraphics[width=0.9\textwidth]{./figuresSynapse/Barroso-Flores+etal+Fig5}
\end{center}
\end{frame}


% oooooooooooooooooooooooooooooo
\begin{frame}
\frametitle{Modelo simple de facilitación-depresión}
\tikzstyle{na} = [baseline=-.5ex]

{\textcolor{green}{Facilitación}: Atribuida principalmente a incrementos
  en la probabilidad de liberación de vesículas,  debido a
  incrementos de calcio residual en la terminal}

\vspace{10pt}
{\textcolor{red}{Depresión}: Atribuida principalmente a decrementos
  en el número de vesículas listas para liberar neurotransmisor
  (VLLs). 
}

\vspace{10pt}
\uncover<2->{Proxy de plasticidad (Liley and North (1953), De Robertis
  and Bennet (1955), Tsodyks \& Markram (1998), etc): 

\begin{equation*}
  \tikz[baseline]{\node[fill=blue!60,anchor=base] (t1){$\eta$};}
  = 
  \tikz[baseline]{\node[fill=red!60,anchor=base] (t2){$x$};}
  \tikz[baseline]{\node[fill=green!60,anchor=base] (t3){$p$};}
\end{equation*}
}
\begin{itemize}[<+-| alert@+>]
  \item<3-> Proporción máxima de vesículas liberadas
    \tikz[na] \node[coordinate] (n1) {};
  \item<4-> Ocupación de la zona de VLLs
    \tikz[na] \node[coordinate] (n2) {};
  \item<5-> Probabilidad de liberación
    \tikz[na]\node [coordinate] (n3) {};
\end{itemize}
\begin{tikzpicture}[overlay]
  \path[->]<3-> (n1) edge [out=0, in=1800] (t1);
  \path[->]<4-> (n2) edge [out=0, in=-135] (t2);
  \path[->]<5-> (n3) edge [out=0, in=-90] (t3);
\end{tikzpicture}
\source{Herrera-Valdez, Barroso}

\end{frame}


% oooooooooooooooooooooooooooooo
\begin{frame}
  \frametitle{Modelo simple de facilitación-depresión 2}
  \tikzstyle{na} = [baseline=-.5ex]
  
  \begin{small}
    Asuma que una (inter)neurona dispara potenciales de acción en
    tiempos  $t_0,...,t_n$. La dinámica de liberación de vesículas en
    la terminal está dada por
    \begin{equation*}
      \partial_t x = 
        \tikz[baseline]{\node[anchor=base] (t0){$x^{k_x}$};} 
        \frac{        \lrRound{
          \tikz[baseline]{\node[anchor=base] (t1){$x_{\infty}$};} -x
        }
      }
      {
        \tikz[baseline]{\node[anchor=base] (t2){$\tau_{r}$};}
      } 
      - p x \sum_{k=0}^n \delta \lrRound{t-t_k}, 
      \qquad 
      \partial_t p 
      =  p^{k_p}
      \frac{   
        \tikz[baseline]{\node[anchor=base] (t4){$p_{\infty}$};} 
        -
        \tikz[baseline]{\node[anchor=base] (t3){$p$};} 
      }
      {\tikz[baseline]{\node[anchor=base] (t5) {$\tau_{p}$};}} 
      + \tikz[baseline]{\node[anchor=base] (t6){$h_k$};}
      \lrRound{1-p} 
      \sum_{k=1}^n \delta \lrRound{t-t_k}, 
    \end{equation*}
  \end{small}
  \begin{minipage}{0.5\textwidth}
    \begin{small}
      % donde $h_k$ depende de $t_{k} -t_{k-1}$ para $k=1,3,...,n$ en
      % una función de la forma
      % \begin{equation}
      %   h_k= \frac{1}{1+ \frac{t_k - t_{k-1}}{m}} 
      % \end{equation}
      
      \begin{itemize}[<+-| alert@+>]
      \item<2-> Proporción de ocupación de sitios de liberación
        inmediata (dinámica logística)
        \tikz[na] \node[coordinate] (n0) {};
      \item<3-> Estado estable de ocupación
        \tikz[na] \node[coordinate] (n1) {};
      \item<4-> Constante de tiempo de reocupación
        \tikz[na] \node[coordinate] (n2) {};
      \end{itemize}
    
    \end{small}
  \end{minipage}%
  \begin{minipage}{0.5\textwidth}
    \begin{small}
      \begin{itemize}[<+-| alert@+>]
      \item<5-> Probabilidad de liberación
        \tikz[na] \node[coordinate] (n3) {};
      \item<6-> Estado estable de $p$
        \tikz[na] \node[coordinate] (n4) {};
      \item<7-> Constante de tiempo de $p$
        \tikz[na] \node[coordinate] (n5) {};
      \item<8-> Factor de facilitación
        \tikz[na]\node [coordinate] (n6) {};
      \end{itemize}
    \end{small}
  \end{minipage}

  \begin{tikzpicture}[overlay]
    \path[->]<2> (n0) edge [out=0, in=90] (t0) ;
    \path[->]<3> (n1) edge [out=0, in=90] (t1);
    \path[->]<4> (n2) edge [out=0, in=-45] (t2);
    \path[->]<5> (n3) edge [out=0, in=0] (t3);
    \path[->]<6> (n4) edge [out=0, in=180] (t4);
    \path[->]<7> (n5) edge [out=0, in=0] (t5);
    \path[->]<8> (n6) edge [out=0, in=90] (t6);
  \end{tikzpicture}

  \source{Barroso-Flores, Herrera-Valdez, et al. 2015; Herrera-Valdez, Barroso, Olicón, Bargas, \textit{en preparación}.}
\end{frame}



% __________________________________________________
%
\section{Dinámica de facilitación y depresión por separado}
% __________________________________________________
%

% oooooooooooooooooooooooooooooo
\titledSlide{Análisis cualitativo de facilitación}{
  \begin{minipage}{0.6\textwidth}
    \begin{center}
      \includegraphics[width=0.9\textwidth]{figuresDynamicalSystems/pLinearPPWPulse1}
    \end{center}
  \end{minipage}%
  \begin{minipage}{0.4\textwidth}
    Punto fijo con o sin pulso:
    \begin{eqnarray*}
      0 =  \partial_t p &=&  \frac{p_{\infty}-p}{\tau_p} + h (1-p) \delta(t-t_i)
      \\
      &=&  \underbrace{\lrRound{ \frac{p_{\infty}}{\tau_p}
          +h}}_{\text{cambio positivo en $p$ para $p=0$}} 
      -p \underbrace{\lrRound{\frac{1}{\tau_p} +h}}_{\text{Tasa de
          cambio}} 
    \end{eqnarray*}
    \uncover<2->{
    Por lo que el punto fijo satisface:
    \begin{eqnarray*}
      p_* &=& \frac
      {\lrRound{ p_{\infty} + \tau_p h}}
      {\lrRound{1 +\tau_p h}}
    \end{eqnarray*}
  }
  \end{minipage}
\begin{small}
  \uncover<3->{
  \bigskip
  Cuando hay un potencial de acción, 
  \begin{enumerate}
  \item Con o sin pulso, la pendiente es negativa, el punto fijo es
    atractor. 
  \item El punto fijo de $p$ se mueve un poco hacia la
    derecha. I.e. la probabilidad de liberación  ``asintótica'' crece durante el pulso. 
  \item El cambio en $p$ es más rápido. 
  \end{enumerate}
}
\end{small}
}

% oooooooooooooooooooooooooooooo
\titledSlide{Análisis cualitativo de depresión}{
  \begin{minipage}{0.6\textwidth}
    \begin{center}
      \only<1>{\includegraphics[width=0.9\textwidth]{figuresDynamicalSystems/xLogisticPPWPulse1}}
      \only<2-3>{\includegraphics[width=0.9\textwidth]{figuresDynamicalSystems/xLogisticPPWPulse2}}
      \only<4->{\includegraphics[width=0.9\textwidth]{figuresDynamicalSystems/xLogisticPPWPulse3}}
    \end{center}
  \end{minipage}%
  \begin{minipage}{0.4\textwidth}
    Punto fijo:
    \begin{eqnarray*}
      0 =  \partial_t x &=&  x \frac{x_{\infty}-x}{\tau_x} - px \alpha\lrRound{t-t_k}
      %\\ &=&  x \lrRound{ \frac{x_{\infty} -x}{\tau_x} - p \alpha\lrRound{t-t_k}}
      \\ &\downarrow& \\
      0 &=&  x \lrRound{ \frac{x_{\infty}- p \tau_x \alpha \lrRound{t-t_k} -x}{\tau_x} }.
    \end{eqnarray*}
    Hay entonces dos puntos fijos: 
    \begin{equation*}
      x_{1*} = 0
      \quad 
      x_{2*} = x_{\infty}- p \tau_x \alpha \lrRound{t-t_k}
    \end{equation*}
    Por tanto, la ocupación se aleja de 0, y se acerca a $x_{2*}$. La
    tasa de cambio primero acelera y
    después desacelera.
  \end{minipage}
  
  \begin{enumerate}
  \item<2-> La pendiente es negativa sólo para algunos valores de $\tau_x$,
    así que el punto fijo es  estable sólo en esos casos. 
  \item<3-> Cuando hay un potencial de acción, 
    \begin{enumerate}
    \item<4->El punto fijo $x_{2*}=x_{\infty}- p \tau_x>x_{\infty}$ es
      atractor cuando $x_{\infty}/p > \tau_x$. Si $\tau_x$ sobrepasa a
      $x_{\infty}/p$ hay una bifurcación ``instantánea'' (no hay
      atractor) que se puede entender en términos de la constante de
      evolución $x_{\infty}/\tau_x$: $x$ tiene un valor atractor si
      $x_{\infty}/\tau_x>p$.
    \end{enumerate}
  \end{enumerate}
}



% oooooooooooooooooooooooooooooo
\titledSlide{Soluciones analíticas y predicción de pulsos}{

\begin{minipage}{0.5\textwidth}
\begin{small}
Consideremos la versión del sistema definida como
\begin{eqnarray*}
\partial_t x &=& \frac{x\lrRound{1-x}}{\tau_{r}} 
-  p x \alpha\lrRound{t-t_i} ,
\\
\partial_t p &=&\frac{p_{\infty}-p}{\tau_{f}} 
+ h_i \lrRound{1-p}   \textcolor{blue}{\alpha\lrRound{t-t_i}},
\label{eq:prelease}
\\
\textcolor{blue}{\alpha \lrRound{t-t_k}} &=& 1 \textrm{ si $t=t_{k}$, 0 si $t\neq t_k$}.
\label{eq:prelease}
\end{eqnarray*}

\uncover<2->{
Sean $(x_k,p_k) = (x(t_k),p(t_k))$ los estados del sistema 
al tiempo del $k$esimo potencial de acción.  

La dinámica de $(x,p)$ entre pulsos $t \in (t_i, t_{i+1})$ satisface que}
\begin{eqnarray*}
\uncover<2->{
p(t) 
&=&  p_{\infty} - \lrRound{p_{\infty}- p_i}
   \exp\lrRound{-\frac{(t-t_i)}{\tau_p}}. }
\label{pLinearSolution}
\\
\uncover<3->{
x(t) 
&=& \frac{x_{\infty} {x_i}}{{x_i} + \lrRound{x_{\infty}- x_i}
\exp\lrRound{- (t-t_i) x_{\infty} / \tau_x }},}
\label{xLogisticSolution}
\end{eqnarray*}


\uncover<4->{
Al tiempo del $(k+1)$ésimo potencial de acción en $t=t_{k+1}$, el
estado $(x,p)$ cambia al valor
\begin{eqnarray*}
x_{k+1} 
&=& \textcolor{red}{\lrRound{1-p_{k+1}}} \frac{x_{\infty} {x_{k}} }{{x_{k}} + \lrRound{x_{\infty}- x_{k}}
\exp\lrRound{- (t_{k+1}-t_{k}) x_{\infty} / \tau_x }}
\\
p_{k+1} 
&=& \textcolor{green}{\lrRound{1+p_{k+1} h_{k+1}}} \lrSquare{{p_{k}} + \lrRound{p_{\infty}- p_{k}}
\exp\lrRound{- (t_{k+1}-t_{k})  / \tau_p }} 
\end{eqnarray*}

Siguiendo esta relación de recurrencia, es posible escribir una
expresión explícita para $(x,p)$ en cualquier tiempo $t$. 
}
\end{small}

\end{minipage}%
\begin{minipage}{0.5\textwidth}
\begin{center}
  \uncover<2->{\includegraphics[width=0.9\textwidth]{figuresDynamicalSystems/pSolutionWPulse1}}
  \uncover<3->{\includegraphics[width=0.9\textwidth]{figuresDynamicalSystems/xLogisticSolutionWPulse1}}
\end{center}

\end{minipage}
}


% __________________________________________________
%
\section{Simulaciones con distintas dinámicas}
% __________________________________________________
% oooooooooooooooooooooooooooooo
\titledSlide{Simulaciones con pulsos contínuos}{
\begin{minipage}{0.5\textwidth}
\begin{small}
Consideremos la versión del sistema definida con pulsos contínuos
\begin{eqnarray*}
\partial_t x &=& \frac{x\lrRound{1-x}}{\tau_{r}} 
-   \sum_{i=1}^n p x \alpha_x \lrRound{t-t_i} ,
\\
\partial_t p &=&\frac{p_{\infty}-p}{\tau_{f}} 
+\lrRound{1-p} \sum_{i=1}^n h_i    \textcolor{blue}{\alpha_p \lrRound{t-t_i}},
\label{eq:prelease}
\\
\textcolor{blue}{\alpha_j \lrRound{t-t_k}} &=& \frac{t-t_k}{\tau}
\exp\lrRound{1-\frac{t-t_k}{\tau}} \textrm{ si $t\geq t_{k}$, 0 si $t<t_k$}. 
\label{eq:prelease}
\end{eqnarray*}
$j \in \lrSet{x,p}$, 
$k=1,...,n$ para $\alpha_p$ y $k=0,...,n$ para $\alpha_x$. 
\end{small}
\end{minipage}%
\begin{minipage}{0.5\textwidth}
\begin{center}
\includegraphics[width=0.9\textwidth]{figuresDynamicalSystems/alphaFunctionTau1.png}
\end{center}
\end{minipage}

% \begin{center}
%   \includegraphics[width=0.9\textwidth]{figuresSynapse/simulationFacilitationXP.png}
% \end{center}
}

% oooooooooooooooooooooooooooooo
\begin{frame}
\frametitle{Modelo simple de facilitación-depresión}
\begin{center}
\includegraphics[height=0.9\textheight]{./figuresSynapse/Barroso-Flores+etal+Fig6}
\end{center}
\end{frame}



% oooooooooooooooooooooooooooooo
\titledSlide{Transición entre perfiles}{
\begin{center}
\only<1>{\includegraphics[width=1.1\textwidth]{figuresSynapse/taux=1_taup[1-90]_h=0dot1_xa=0dot9_pa=0dot3.png}}\only<2>{\includegraphics[width=1.1\textwidth]{figuresSynapse/taux=10_taup[1-90]_h=0dot1_xa=0dot9_pa=0dot3.png}}\only<3>{\includegraphics[width=1.1\textwidth]{figuresSynapse/taux=20_taup[1-90]_h=0dot1_xa=0dot9_pa=0dot3.png}}\only<4>{\includegraphics[width=1.1\textwidth]{figuresSynapse/taux=30_taup[1-90]_h=0dot1_xa=0dot9_pa=0dot3.png}}
\end{center}
}

% oooooooooooooooooooooooooooooo
\titledSlide{Perfiles bifásicos}{
\begin{center}
\only<1>{\includegraphics[width=1.1\textwidth]{figuresSynapse/taux=30_taup[1-285]_h=0dot1_xa=1_pa=0dot1}}\only<2>{\includegraphics[width=1.1\textwidth]{figuresSynapse/taux=45_taup[1-285]_h=0dot1_xa=1_pa=0dot1}}
\end{center}
}



% __________________________________________________
%
\section{Conclusiones}
% __________________________________________________

% oooooooooooooooooooooooooooooo
\titledSlide{Resúmen}{
\begin{itemize}
\item<1-> El análisis geométrico permite establecer nociones precisas
  sobre la dependencia de $xp$ en la interacción entre la probabilidad de liberación, la
  ocupación, el cambio en $p$ por pulso, y el intervalo inter-spike. 
\item<2-> El modelo propuesto reproduce los perfiles de facilitación,
  depresión, y bifásico.
\item<3-> El balance entre las constantes de tiempo de ocupación y
  probabilidad de liberación determina en muchos casos si hay
  facilitación o depresión. 
\item<4-> Sin embargo, a veces el cambio en la
  probabilidad dado un potencial de acción determina si hay
  facilitación o depresión  
\end{itemize}
}


% oooooooooooooooooooooooooooooo
\titledSlide{Resúmen}{
Gracias.
}


% __________________________________________________
%
\section{Cálculos en más detalle}
% __________________________________________________
% oooooooooooooooooooooooooooooooooooooooo
\begin{frame}
\frametitle{Soluciones analíticas y predicción de pulsos: $p$}
\begin{small}
In general, for $t \in (t_{m},t_{m+1})$
\begin{eqnarray}
p(t)
&=& 
p_{\infty} + (1-p_{\infty}) 
\sum_{k=1}^{m} 
h_k \lrSquare{
\prod_{l=k+1}^{m}  
\lrRound{1- h_{l}}} 
\exp \lrRound{-\frac{t-t_k}{\tau_p}}. 
\label{eq:pBetweenAP}
\end{eqnarray}
and at  $t=t_{m+1}$, 
\begin{eqnarray}
p(t_{m+1})
&=&h_{m+1} + \lrRound{1-h_{m+1}}
\lrSet{
 p_{\infty} + \lrRound{1-p_{\infty}} 
\sum_{k=1}^{m} 
h_k \lrSquare{
\prod_{l=k+1}^{m}  
\lrRound{1- h_{l}}} 
\exp \lrRound{-\frac{t_{m+1}-t_k}{\tau_p}}
}. 
\label{eq:pAP}
\end{eqnarray}

\textbf{Simplification.} 
Assume that $h_k =h$ for
$k=1,...,n$.  Then, between action potentials occuring at times $t_m$
and $t_{m+1}$,  equation \eqref{eq:pBetweenAP} simplifies to
\begin{eqnarray*}
p(t)
&=& 
p_{\infty} + (1-p_{\infty}) 
\sum_{k=1}^{m} 
h \lrSquare{
\prod_{l=k+1}^{m}  
\lrRound{1- h}} 
\exp \lrRound{-\frac{t-t_k}{\tau_p}}. 
\nonumber \\ 
&=& 
p_{\infty} + (1-p_{\infty}) 
h \sum_{k=1}^{m} 
\lrRound{1- h}^{m-k} 
\exp \lrRound{-\frac{t-t_k}{\tau_p}}. 
\end{eqnarray*}
At $t=t_{m+1}$, the probability of release is
\begin{eqnarray*}
p(t_{m+1})
&=&
p_{\infty} + (1-p_{\infty}) 
h \sum_{k=1}^{m} 
\lrRound{1- h}^{m-k} 
\exp \lrRound{-\frac{t_{m+1}-t_k}{\tau_p}}. 
\end{eqnarray*}
\end{small}
\end{frame}


% oooooooooooooooooooooooooooooooooooooooo
\begin{frame}
\frametitle{Simplificación 2: $p(t)$ para intervalos entre PAs constantes}
\begin{small}
If the interspike intervals are such that $\delta = t_{m} - t_{m-1}$
for all $m \in \lrSet{1,...,n}$, then  the equation \eqref{eq:pBetweenAP}
simplifies further to
\begin{eqnarray*}
p(t)
&=& 
p_{\infty} + (1-p_{\infty}) 
 \exp \lrRound{-\frac{t-t_{m}}{\tau_p}}
F(h,\delta,m)
\end{eqnarray*}
where 
\begin{eqnarray*}
F(h,\delta, m) 
&=& 
h 
\frac
{ \lrRound{1- h}^{m+1}
\exp \lrRound{- m \delta / \tau_p}
- {\exp \lrRound{  \delta / \tau_p}}
}
{\lrRound{1- h} - \exp \lrRound{ \delta / \tau_p}}
\end{eqnarray*}
At $t=t_{m+1}$, 
\begin{eqnarray*}
p(t_{m+1})
&=&
p_{\infty} + (1-p_{\infty}) 
 \exp \lrRound{-\frac{t_{m+1}-t_{m}}{\tau_p}}
F(h,\delta,m)
\end{eqnarray*}
\end{small}
\end{frame}


% oooooooooooooooooooooooooooooooooooooooo
\begin{frame}
\frametitle{Análisis de ocupación en respuesta a pulsos}

En general, para el $n$ésimo pulso,
\begin{eqnarray}
x_n
&=& 
\frac{
x_{\infty} \lrRound{1-p}^{n+1} 
}
{
\lrRound{1-p}^{n} 
+ p
\sum_{i=1}^{n} \lrRound{1-p}^{n-i} \exp\lrSquare{- (t_n-t_{n-i}) r_x }
}
\label{tnOccup1}
\\
&=& 
\frac{
x_{\infty} \lrRound{1-p}
}{
1
+ p
\sum_{i=1}^{n} \lrRound{1-p}^{-i} \exp\lrSquare{- (t_n-t_{n-i}) r_x }
}
\label{tnOccup2}
\\
&=& 
\frac{
x_{\infty} \lrRound{1-p}
}{
1
+ p \exp\lrSquare{- t_n r_x}
\sum_{i=1}^{n} \lrRound{1-p}^{-i} \exp\lrSquare{t_{n-i} r_x }
}
\label{tnOccup3}
\end{eqnarray}
\end{frame}


\end{document}
